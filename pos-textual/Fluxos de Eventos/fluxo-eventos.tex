\documentclass[a4paper,12pt]{article}
\usepackage[utf8]{inputenc}

\begin{document}

\begin{center}
\textbf{\huge{Fluxos de eventos}}
\end{center}

%criar página
\vspace{0.7cm}
\leftline{ \textbf{1. Caso de uso}: Criar página.}

\noindent \textit{Pré-condição}: selecionar o botão de salvar.

\noindent \textit{Fluxo principal}:

\begin{enumerate}
    \item selecionar no menu o item ``Conteúdo'';
    \item selecionar no menu o item ``Adicionar item'';
    \item selecionar o item ``Página'';
    \item preencher obrigatoriamente o título; e
    \item selecionar a opção ``Salvar''.
\end{enumerate}

\noindent \textit{Fluxo alternativo}: Cancelamento do cadastro.

\noindent \textit{Pré-condição}: usuário acionou o botão ``Cancelar''.

\noindent \textit{Etapas}:

\begin{enumerate}
    \item selecionar o botão ``Cancelar''; e
    \item o sistema retorna a mensagem: ``Adicionar Novo Item operação cancelada''.
\end{enumerate}

\noindent \textit{Fluxo alternativo}: Não preenchimento do título.

\noindent \textit{Pré-condição}: usuário acionou o botão ``Salvar'' sem ter inserido o título da página.

\noindent \textit{Etapas}:

\begin{enumerate}
    \item selecionar o botão ``Salvar'';
    \item o sistema retorna a mensagem: ``Existem alguns erros''; e
    \item o campo de título apresenta a seguinte mensagem: ``Dado obrigatório não informado''.
\end{enumerate}



%alterar página
\vspace{0.7cm}
\leftline{ \textbf{2. Caso de uso}: Alterar página.}

\noindent \textit{Pré-condição}: selecionar a página.

\noindent \textit{Fluxo principal}:

\begin{enumerate}
    \item selecionar no menu o item ``Conteúdo'';
    \item selecionar a página;
    \item selecionar no menu o item ``Edição''; e
    \item após realizar as alterações, ir em salvar.
\end{enumerate}

\noindent \textit{Fluxo alternativo}: Cancelamento de alterações.

\noindent \textit{Pré-condição}: usuário acionou o botão ``Cancelar''.

\noindent \textit{Etapas}:

\begin{enumerate}
    \item selecionar o botão ``Cancelar''; e
    \item o sistema retorna a mensagem: ``Edição cancelada''.
\end{enumerate}


%excluir página
\vspace{0.7cm}
\leftline{ \textbf{3. Caso de uso}: Excluir página.}

\noindent \textit{Pré-condição}: selecionar a opção de excluir a página.

\noindent \textit{Fluxo principal}:

\begin{enumerate}
    \item selecionar no menu o item ``Conteúdo'';
    \item selecionar a página;
    \item selecionar no ícone representado por uma lixeira no canto superior da tela;
    \item o sistema emitirá a mensagem: ``Você tem certeza de que deseja excluir os itens selecionados?''; e
    \item selecionar o botão de confirmação da ação.
\end{enumerate}

\noindent \textit{Fluxo alternativo}: Cancelamento da exclusão.

\noindent \textit{Pré-condição}: usuário acionou o botão ``Não''.

\noindent \textit{Etapas}:

\begin{enumerate}
    \item selecionar o botão ``Não''.
\end{enumerate}


%cadastrar notícia
\vspace{0.7cm}
\leftline{ \textbf{4. Caso de uso}: Publicar notícia.}

\noindent \textit{Pré-condição}: selecionar a opção de salvar a notícia.

\noindent \textit{Fluxo principal}:

\begin{enumerate}
    \item selecionar no menu o item ``Conteúdo'';
    \item selecionar a pasta notícias;
    \item selecionar a pasta do ano da notícia;
    \item selecionar no menu o item ``Adicionar item'';
    \item selecionar o item ``Notícia'';
    \item preencher obrigatoriamente o título;
    \item selecionar a opção ``Salvar'';
    \item selecionar no menu o item ``Estado''; e
    \item selecionar a opção ``Publicar''.
\end{enumerate}

\noindent \textit{Fluxo alternativo}: Cancelamento do cadastro.

\noindent \textit{Pré-condição}:  usuário acionou o botão ``Cancelar''.

\noindent \textit{Etapas}:

\begin{enumerate}
    \item selecionar o botão ``Cancelar''; e
    \item o sistema retorna a mensagem: ``Adicionar Novo Item operação cancelada''.
\end{enumerate}

\noindent \textit{Fluxo alternativo}: Não preenchimento do título.

\noindent \textit{Pré-condição}: usuário acionou o botão ``Salvar'' sem ter inserido o título da notícia.

\noindent \textit{Etapas}:

\begin{enumerate}
    \item selecionar o botão ``Salvar'';
    \item o sistema retorna a mensagem: ``Existem alguns erros''; e
    \item o campo de título apresenta a seguinte mensagem: ``Dado obrigatório não informado''.
\end{enumerate}




%editar notícia
\vspace{0.7cm}
\leftline{ \textbf{5. Caso de uso}: Alterar notícia.}

\noindent \textit{Pré-condição}: selecionar a opção de editar a notícia.

\noindent \textit{Fluxo principal}:

\begin{enumerate}
    \item selecionar no menu o item ``Conteúdo'';
    \item selecionar a pasta notícias;
    \item selecionar a pasta do ano da notícia;
    \item selecionar a notícia; e
    \item selecionar no menu o item ``Edição''.
\end{enumerate}

\noindent \textit{Fluxo alternativo}: Cancelamento da edição.

\noindent \textit{Pré-condição}: usuário acionou o botão ``Cancelar''.

\noindent \textit{Etapas}:

\begin{enumerate}
    \item selecionar o botão ``Cancelar''; e
    \item o sistema retorna a mensagem: ``Edição cancelada''.
\end{enumerate}

\noindent \textit{Fluxo alternativo}: Não preenchimento do título.

\noindent \textit{Pré-condição}: usuário acionou o botão ``Salvar'' sem ter inserido o título da notícia.

\noindent \textit{Etapas}:

\begin{enumerate}
    \item selecionar o botão ``Salvar'';
    \item o sistema retorna a mensagem: ``Existem alguns erros''; e
    \item o campo de título apresenta a seguinte mensagem: ``Dado obrigatório não informado''.
\end{enumerate}




%excluir notícia
\vspace{0.7cm}
\leftline{ \textbf{6. Caso de uso}: Excluir notícia.}

\noindent \textit{Pré-condição}: selecionar a opção de excluir a notícia.

\noindent \textit{Fluxo principal}:

\begin{enumerate}
    \item selecionar no menu o item ``Conteúdo'';
    \item selecionar a pasta notícias;
    \item selecionar a pasta do ano da notícia;
    \item selecionar a notícia;
    \item selecionar no menu o item ``Ações'';
    \item ir em ``Excluir'';
    \item o sistema emitirá a mensagem: ``Você realmente quer apagar este item?''; e
    \item selecionar a opção de excluir.
\end{enumerate}

\noindent \textit{Fluxo alternativo}: Cancelamento da exclusão.

\noindent \textit{Pré-condição}: usuário acionou o botão ``Cancelar''.

\noindent \textit{Etapas}:

\begin{enumerate}
    \item selecionar o botão ``Cancelar''.
\end{enumerate}


%cadastrar evento
\vspace{0.7cm}
\leftline{ \textbf{7. Caso de uso}: Publicar evento.}

\noindent \textit{Pré-condição}: selecionar a opção de salvar o evento.

\noindent \textit{Fluxo principal}:

\begin{enumerate}
    \item selecionar no menu o item ``Conteúdo'';
    \item selecionar a pasta eventos;
    \item selecionar a pasta do ano do evento;
    \item selecionar no menu o item ``Adicionar item'';
    \item selecionar o item ``Evento'';
    \item preencher obrigatoriamente o título;
    \item selecionar a opção ``Salvar'';
    \item selecionar no menu o item ``Estado''; e
    \item selecionar a opção ``Publicar''.
\end{enumerate}

\noindent \textit{Fluxo alternativo}: Cancelamento do cadastro.

\noindent \textit{Pré-condição}:  usuário acionou o botão ``Cancelar''.

\noindent \textit{Etapas}:

\begin{enumerate}
    \item selecionar o botão ``Cancelar''; e
    \item o sistema retorna a mensagem: ``Adicionar Novo Item operação cancelada''.
\end{enumerate}

\noindent \textit{Fluxo alternativo}: Não preenchimento do título.

\noindent \textit{Pré-condição}: usuário acionou o botão ``Salvar'' sem ter inserido o título do evento.

\noindent \textit{Etapas}:

\begin{enumerate}
    \item selecionar o botão ``Salvar'';
    \item o sistema retorna a mensagem: ``Existem alguns erros''; e
    \item o campo de título apresenta a seguinte mensagem: ``Dado obrigatório não informado''.
\end{enumerate}



%editar evento
\vspace{0.7cm}
\leftline{ \textbf{8. Caso de uso}: Alterar evento.}

\noindent \textit{Pré-condição}: selecionar a opção de editar o evento.

\noindent \textit{Fluxo principal}:

\begin{enumerate}
    \item selecionar no menu o item ``Conteúdo'';
    \item selecionar a pasta eventos;
    \item selecionar a pasta do ano do evento;
    \item selecionar o evento; e
    \item selecionar no menu o item ``Edição''.
\end{enumerate}

\noindent \textit{Fluxo alternativo}: Cancelamento da edição.

\noindent \textit{Pré-condição}: usuário acionou o botão ``Cancelar''.

\noindent \textit{Etapas}:

\begin{enumerate}
    \item selecionar o botão ``Cancelar''; e
    \item o sistema retorna a mensagem: ``Edição cancelada''.
\end{enumerate}

\noindent \textit{Fluxo alternativo}: Não preenchimento do título.

\noindent \textit{Pré-condição}: usuário acionou o botão ``Salvar'' sem ter inserido o título do evento.

\noindent \textit{Etapas}:

\begin{enumerate}
    \item selecionar o botão ``Salvar'';
    \item o sistema retorna a mensagem: ``Existem alguns erros''; e
    \item o campo de título apresenta a seguinte mensagem: ``Dado obrigatório não informado''.
\end{enumerate}



%excluir evento
\vspace{0.7cm}
\leftline{ \textbf{9. Caso de uso}: Excluir evento.}

\noindent \textit{Pré-condição}: selecionar a opção de excluir o evento.

\noindent \textit{Fluxo principal}:

\begin{enumerate}
    \item selecionar no menu o item ``Conteúdo'';
    \item selecionar a pasta eventos;
    \item selecionar a pasta do ano do evento;
    \item selecionar o evento;
    \item selecionar no menu o item ``Ações'';
    \item ir em ``Excluir'';
    \item o sistema emitirá a mensagem: ``Você realmente quer apagar este item?''; e
    \item selecionar a opção de excluir.
\end{enumerate}

\noindent \textit{Fluxo alternativo}: Cancelamento da exclusão.

\noindent \textit{Pré-condição}: usuário acionou o botão ``Cancelar''.

\noindent \textit{Etapas}:

\begin{enumerate}
    \item selecionar o botão ``Cancelar''.
\end{enumerate}



%cadastrar arquivo
\vspace{0.7cm}
\leftline{ \textbf{10. Caso de uso}: Cadastrar arquivo.}

\noindent \textit{Pré-condição}: selecionar a opção de adicionar arquivo ou adicionar imagem.

\noindent \textit{Fluxo principal}:

\begin{enumerate}
    \item selecionar no menu o item ``Conteúdo'';
    \item selecionar a pasta onde deseja adicionar o arquivo;
    \item selecionar no menu o item ``Adicionar arquivo'';
    \item anexar obrigatoriamente o arquivo;
    \item selecionar a opção ``Salvar'';
    \item sistema retorna a mensagem: ``item criado''.
\end{enumerate}

\noindent \textit{Fluxo alternativo}: Cancelamento do cadastro.

\noindent \textit{Pré-condição}:  usuário acionou o botão ``Cancelar''.

\noindent \textit{Etapas}:

\begin{enumerate}
    \item selecionar o botão ``Cancelar''; e
    \item o sistema leva o usuário até o diretório onde estava.
\end{enumerate}



%editar arquivo
\vspace{0.7cm}
\leftline{ \textbf{11. Caso de uso}: Alterar arquivo.}

\noindent \textit{Pré-condição}: selecionar a opção de editar o arquivo.

\noindent \textit{Fluxo principal}:

\begin{enumerate}
    \item selecionar no menu o item ``Conteúdo'';
    \item selecionar a pasta do arquivo;
    \item selecionar o arquivo; e
    \item selecionar no menu o item ``Edição''.
\end{enumerate}

\noindent \textit{Fluxo alternativo}: Cancelamento da edição.

\noindent \textit{Pré-condição}: usuário acionou o botão ``Cancelar''.

\noindent \textit{Etapas}:

\begin{enumerate}
    \item selecionar o botão ``Cancelar''; e
    \item o sistema retorna a mensagem: ``Edição cancelada''.
\end{enumerate}


%excluir arquivo
\vspace{0.7cm}
\leftline{ \textbf{11. Caso de uso}: Excluir arquivo.}

\noindent \textit{Pré-condição}: selecionar a opção de excluir o arquivo.

\noindent \textit{Fluxo principal}:

\begin{enumerate}
    \item selecionar no menu o item ``Conteúdo'';
    \item selecionar a pasta do arquivo;
    \item selecionar o arquivo;
    \item selecionar no menu o item ``Ações'';
    \item ir em ``Excluir'';
    \item o sistema emitirá a mensagem: ``Você realmente quer apagar este item?''; e
    \item selecionar a opção de excluir.
\end{enumerate}

\noindent \textit{Fluxo alternativo}: Cancelamento da exclusão.

\noindent \textit{Pré-condição}: usuário acionou o botão ``Cancelar''.

\noindent \textit{Etapas}:

\begin{enumerate}
    \item selecionar o botão ``Cancelar''.
\end{enumerate}



%Gerenciar usuários
\vspace{0.7cm}
\leftline{ \textbf{12. Caso de uso}: Consultar usuários.}

\noindent \textit{Pré-condição}: estar na página de usuários.

\noindent \textit{Fluxo principal}:

\begin{enumerate}
    \item selecionar no menu o item ``Configuração de site''; e
    \item selecionar a opção ``Usuários e grupos''.
\end{enumerate}

\noindent \textit{Fluxo alternativo}: não se aplica.



%cadastrar usuário
\vspace{0.7cm}
\leftline{ \textbf{13. Caso de uso}: Cadastrar usuário.}

\noindent \textit{Pré-condição}: selecionar o botão ``Registrar''.

\noindent \textit{Fluxo principal}:

\begin{enumerate}
    \item selecionar no menu o item ``Configuração de site'';
    \item selecionar a opção ``Usuários e grupos'';
    \item selecionar o botão ``Adicionar Novo Usuário'';
    \item informar, obrigatoriamente, e-mail e nome de usuário; e
    \item selecionar o botão de registrar o usuário.
\end{enumerate}

\noindent \textit{Fluxo alternativo}: Não preenchimento de e-mail ou nome de usuário.

\noindent \textit{Pré-condição}: usuário acionou o botão ``Registrar'' sem ter inserido o e-mail ou nome de usuário.

\noindent \textit{Etapas}:

\begin{enumerate}
    \item selecionar o botão ``Registrar'';
    \item o sistema retorna a mensagem: ``Houve erros''; e
    \item os campos de e-mail e usuário apresentam a seguinte mensagem: ``Dado obrigatório não informado''.
\end{enumerate}


%alterar usuário
\vspace{0.7cm}
\leftline{ \textbf{14. Caso de uso}: Alterar usuário.}

\noindent \textit{Pré-condição}: selecionar o botão ``Salvar''.

\noindent \textit{Fluxo principal}:

\begin{enumerate}
    \item selecionar no menu o item ``Configuração de site'';
    \item selecionar a opção ``Usuários e grupos'';
    \item preencher o campo ``Busca de Usuários'';
    \item selecionar o nome do usuário requisitado;
    \item preencher substituir as informações dos campos necessários; e
    \item acessar o botão de salvar as informações.
\end{enumerate}

\noindent \textit{Fluxo alternativo}: Cancelamento da edição.

\noindent \textit{Pré-condição}: usuário acionou o botão ``Cancelar''.

\noindent \textit{Etapas}:

\begin{enumerate}
    \item selecionar o botão ``Cancelar''; e
    \item sistema exibe a mensagem: ``Alterações canceladas''.
\end{enumerate}

\noindent \textit{Fluxo alternativo}: Não preenchimento do e-mail.

\noindent \textit{Pré-condição}: usuário acionou o botão ``Salvar'' sem ter haver um e-mail inserido.

\noindent \textit{Etapas}:

\begin{enumerate}
    \item o sistema retorna a mensagem: ``Existem alguns erros''; e
    \item o campo de e-mail apresenta a seguinte mensagem: ``Dado obrigatório não informado''.
\end{enumerate}



%remover usuário
\vspace{0.7cm}
\leftline{ \textbf{15. Caso de uso}: Excluir usuário.}

\noindent \textit{Pré-condição}: selecionar o botão ``Salvar''.

\noindent \textit{Fluxo principal}:

\begin{enumerate}
    \item selecionar no menu o item ``Configuração de site'';
    \item selecionar a opção ``Usuários e grupos'';
    \item preencher o campo ``Busca de Usuários'';
    \item selecionar a caixa ``Remover'' na linha onde se encontra o nome do usuário requisitado; e
    \item selecionar o botão de salvar as informações.
\end{enumerate}

\noindent \textit{Fluxo alternativo}: não se aplica.


%enviar mensagem
\vspace{0.7cm}
\leftline{ \textbf{16. Caso de uso}: Enviar mensagem para a associação.}

\noindent \textit{Pré-condição}: selecionar o botão ``Enviar''.

\noindent \textit{Fluxo principal}:

\begin{enumerate}
    \item acessar a aba contato;
    \item inserir as informações para a mensagem; e
    \item selecionar o botão de envio.
\end{enumerate}

\noindent \textit{Fluxo alternativo}: Não preenchimento do nome, e-mail, assunto ou corpo da mensagem.

\noindent \textit{Pré-condição}: usuário acionou o botão ``Enviar'' sem ter preenchido nome, e-mail, assunto ou texto da mensagem.

\noindent \textit{Etapas}:

\begin{enumerate}
    \item os campos obrigatórios vazios apresentam a seguinte mensagem: ``Preencha este campo''.
\end{enumerate}

%SISTEMA DE ASSOCIADOS

%consultar associados
\vspace{0.7cm}
\leftline{ \textbf{17. Caso de uso}: Consultar associados.}

\noindent \textit{Pré-condição}: preencher o campo ``Pesquisar''.

\noindent \textit{Fluxo principal}:

\begin{enumerate}
    \item na página inicial do sistema, inserir alguma informação sobre qualquer atributo do usuário; e
    \item a medida que o usuário digitar, o sistema retornará, em tempo real, os resultados para busca.
\end{enumerate}

\noindent \textit{Fluxo alternativo}: não se aplica.


%consultar associados
\vspace{0.7cm}
\leftline{ \textbf{18. Caso de uso}: Ver detalhes.}

\noindent \textit{Pré-condição}: selecionar o botão ``Exibir detalhes''.

\noindent \textit{Fluxo principal}:

\begin{enumerate}
    \item na página inicial do sistema, selecionar o botão ``Exibir detalhes'' de um associado; e
    \item o sistema retorna as seguintes informações do associado: número de identificação, nome, apelido, rua, número, bairro, número de telefone/celular, e-mail, número do Registro Geral (RG) e número do Cadastro de Pessoas Físicas (CPF).
\end{enumerate}

\noindent \textit{Fluxo alternativo}: não se aplica.


%cadastrar associado
\vspace{0.7cm}
\leftline{ \textbf{19. Caso de uso}: Cadastrar associado.}

\noindent \textit{Pré-condição}: selecionar a opção de registrar o associado.

\noindent \textit{Fluxo principal}:

\begin{enumerate}
    \item acionar o botão ``Adicionar Associado'';
    \item preencher obrigatoriamente nome, e-mail identidade, CPF, telefone, rua, número e bairro; e
    \item selecionar o botão ``Registrar Associado''.
\end{enumerate}

\noindent \textit{Fluxo alternativo}: Cancelamento do cadastro.

\noindent \textit{Pré-condição}: usuário acionou o botão ``Voltar''.

\noindent \textit{Etapas}:

\begin{enumerate}
    \item selecionar o botão ``Voltar''.
\end{enumerate}

\noindent \textit{Fluxo alternativo}: Não preenchimento dos dados obrigatórios.

\noindent \textit{Pré-condição}: usuário acionou o botão ``Salvar'' sem ter inserido os dados obrigatórios.

\noindent \textit{Etapas}:

\begin{enumerate}
    \item selecionar o botão ``Salvar''; e
    \item os campos obrigatórios apresentam a seguinte mensagem: ``Preencha este campo''.
\end{enumerate}


\noindent \textit{Fluxo alternativo}: Endereço de e-mail não é válido.

\noindent \textit{Pré-condição}: usuário solicitou cadastrar associado com um e-mail inválido.

\noindent \textit{Etapas}:

\begin{enumerate}
    \item selecionar o botão ``Salvar''; e
    \item sistema retorna a seguinte mensagem: ``E-mail inválido!''.
\end{enumerate}


\noindent \textit{Fluxo alternativo}: CPF não é válido.

\noindent \textit{Pré-condição}: usuário solicitou cadastrar associado com um CPF inválido.

\noindent \textit{Etapas}:

\begin{enumerate}
    \item selecionar o botão ``Salvar''; e
    \item sistema retorna a seguinte mensagem: ``CPF inválido!''.
\end{enumerate}


\noindent \textit{Fluxo alternativo}: Identidade ou CPF já existente na base de dados.

\noindent \textit{Pré-condição}: usuário solicitou cadastrar associado com um número de identidade ou CPF já existente no banco de dados.

\noindent \textit{Etapas}:

\begin{enumerate}
    \item selecionar o botão ``Salvar''; e
    \item sistema retorna a seguinte mensagem: ``CPF ou identidade já existe no sistema!''.
\end{enumerate}


%editar associado
\vspace{0.7cm}
\leftline{ \textbf{20. Caso de uso}: Alterar associado.}

\noindent \textit{Pré-condição}: selecionar a opção de atualizar o associado.

\noindent \textit{Fluxo principal}:

\begin{enumerate}
    \item selecionar o ícone de edição na linha do registro a ser atualizado;
    \item editar os campos necessários; e
    \item selecionar o botão de atualizar o associado.
\end{enumerate}

\noindent \textit{Fluxo alternativo}: Cancelamento da edição.

\noindent \textit{Pré-condição}: usuário acionou o botão ``Voltar''.

\noindent \textit{Etapas}:

\begin{enumerate}
    \item Selecionar o botão ``Voltar''.
\end{enumerate}

\noindent \textit{Fluxo alternativo}: Não preenchimento dos dados obrigatórios.

\noindent \textit{Pré-condição}: usuário acionou o botão ``Salvar'' sem ter inserido os dados obrigatórios.

\noindent \textit{Etapas}:

\begin{enumerate}
    \item selecionar o botão ``Atualizar Associado''; e
    \item os campos obrigatórios apresentam a seguinte mensagem: ``Preencha este campo''.
\end{enumerate}


\noindent \textit{Fluxo alternativo}: Endereço de e-mail não é válido.

\noindent \textit{Pré-condição}: usuário solicitou atualizar associado com um e-mail inválido.

\noindent \textit{Etapas}:

\begin{enumerate}
    \item selecionar o botão ``Atualizar Associado''; e
    \item sistema retorna a seguinte mensagem: ``E-mail inválido!''.
\end{enumerate}


\noindent \textit{Fluxo alternativo}: CPF não é válido.

\noindent \textit{Pré-condição}: usuário solicitou atualizar associado com um CPF inválido.

\noindent \textit{Etapas}:

\begin{enumerate}
    \item selecionar o botão ``Atualizar Associado''; e
    \item sistema retorna a seguinte mensagem: ``CPF inválido!''.
\end{enumerate}


\noindent \textit{Fluxo alternativo}: Identidade ou CPF já existente na base de dados.

\noindent \textit{Pré-condição}: usuário solicitou atualizar associado com um número de identidade ou CPF já existente no banco de dados.

\noindent \textit{Etapas}:

\begin{enumerate}
    \item selecionar o botão ``Atualizar Associado''; e
    \item sistema retorna a seguinte mensagem: ``CPF ou identidade já existe no sistema!''.
\end{enumerate}



%excluir associado
\vspace{0.7cm}
\leftline{ \textbf{21. Caso de uso}: Excluir associado.}

\noindent \textit{Pré-condição}: selecionar a opção de excluir o associado.

\noindent \textit{Fluxo principal}:

\begin{enumerate}
    \item selecionar o ícone de exclusão na linha do registro a ser atualizado;
    \item o sistema exibirá uma mensagem de confirmação de exclusão; e
    \item selecionar o botão ``OK''.
\end{enumerate}

\noindent \textit{Fluxo alternativo}: Cancelamento da exclusão.

\noindent \textit{Pré-condição}: usuário acionou o botão ``Cancelar''.

\noindent \textit{Etapas}:

\begin{enumerate}
    \item selecionar o botão ``Cancelar''.
\end{enumerate}


%%Implementar e alterar fluxo
%consultar usuário
\vspace{0.7cm}
\leftline{ \textbf{22. Caso de uso}: Consultar usuários.}

\noindent \textit{Pré-condição}: preencher o campo ``Pesquisar''.

\noindent \textit{Fluxo principal}:

\begin{enumerate}
    \item na página inicial do sistema, inserir alguma informação sobre qualquer atributo do usuário; e
    \item a medida que o usuário digitar, o sistema retornará, em tempo real, os resultados para busca.
\end{enumerate}

\noindent \textit{Fluxo alternativo}: não se aplica.


%cadastrar usuário
\vspace{0.7cm}
\leftline{ \textbf{23. Caso de uso}: Cadastrar usuário.}

\noindent \textit{Pré-condição}: selecionar a opção de registrar administrador.

\noindent \textit{Fluxo principal}:

\begin{enumerate}
    \item acionar o botão ``Registrar Administrador'';
    \item preencher nome, e-mail, senha e nível de permissão (secretário ou presidente); e
    \item selecionar o botão ``Registrar Administrador''.
\end{enumerate}

\noindent \textit{Fluxo alternativo}: Cancelamento do cadastro.

\noindent \textit{Pré-condição}: usuário acionou o botão ``Voltar''.

\noindent \textit{Etapas}:

\begin{enumerate}
    \item selecionar o botão ``Voltar''.
\end{enumerate}

\noindent \textit{Fluxo alternativo}: Não preenchimento dos dados obrigatórios.

\noindent \textit{Pré-condição}: usuário acionou o botão ``Salvar'' sem ter inserido os dados obrigatórios.

\noindent \textit{Etapas}:

\begin{enumerate}
    \item selecionar o botão ``Registrar Administrador''; e
    \item os campos obrigatórios apresentam a seguinte mensagem: ``Preencha este campo''.
\end{enumerate}


\noindent \textit{Fluxo alternativo}: Nível de permissão de secretário.

\noindent \textit{Pré-condição}: usuário do tipo ``secretário'' solicitou cadastrar administrador ao sistema.

\noindent \textit{Etapas}:

\begin{enumerate}
    \item selecionar o botão ``Adicionar Administrador''; e
    \item sistema retorna a seguinte mensagem: ``Você não tem permissão para realizar essa operação!''.
\end{enumerate}


\noindent \textit{Fluxo alternativo}: Endereço de e-mail não é válido.

\noindent \textit{Pré-condição}: usuário solicitou cadastrar usuário com um e-mail inválido.

\noindent \textit{Etapas}:

\begin{enumerate}
    \item selecionar o botão ``Registrar Administrador''; e
    \item sistema retorna a seguinte mensagem: ``E-mail inválido!''.
\end{enumerate}


\noindent \textit{Fluxo alternativo}: E-mail já existente na base de dados.

\noindent \textit{Pré-condição}: usuário solicitou cadastrar usuário com um endereço de e-mail já existente no banco de dados.

\noindent \textit{Etapas}:

\begin{enumerate}
    \item selecionar o botão ``Registrar Administrador''; e
    \item sistema retorna a seguinte mensagem: ``E-mail já existe no sistema!''.
\end{enumerate}


%editar usuário
\vspace{0.7cm}
\leftline{ \textbf{24. Caso de uso}: Alterar usuário.}

\noindent \textit{Pré-condição}: selecionar a opção de editar o usuário.

\noindent \textit{Fluxo principal}:

\begin{enumerate}
    \item selecionar o ícone de edição na linha do registro a ser atualizado;
    \item editar os campos necessários; e
    \item selecionar o botão de atualizar o administrador.
\end{enumerate}

\noindent \textit{Fluxo alternativo}: Cancelamento da edição.

\noindent \textit{Pré-condição}: usuário acionou o botão ``Voltar''.

\noindent \textit{Etapas}:

\begin{enumerate}
    \item Selecionar o botão ``Voltar''.
\end{enumerate}

\noindent \textit{Fluxo alternativo}: Não preenchimento dos dados obrigatórios.

\noindent \textit{Pré-condição}: usuário acionou o botão ``Salvar'' sem ter inserido os dados obrigatórios.

\noindent \textit{Etapas}:

\begin{enumerate}
    \item selecionar o botão ``Atualizar Administrador''; e
    \item os campos obrigatórios apresentam a seguinte mensagem: ``Preencha este campo''.
\end{enumerate}



\noindent \textit{Fluxo alternativo}: Secretário tenta editar dados de outro administrador.

\noindent \textit{Pré-condição}: usuário do tipo ``secretário'' solicitou editar outro administrador do sistema.

\noindent \textit{Etapas}:

\begin{enumerate}
    \item selecionar o botão ``Editar''; e
    \item sistema retorna a seguinte mensagem: ``Você não tem permissão para realizar essa operação!''.
\end{enumerate}


\noindent \textit{Fluxo alternativo}: Endereço de e-mail não é válido.

\noindent \textit{Pré-condição}: usuário solicitou alterar administrador com um e-mail inválido.

\noindent \textit{Etapas}:

\begin{enumerate}
    \item selecionar o botão ``Atualizar Administrador'; e
    \item sistema retorna a seguinte mensagem: ``E-mail inválido!''.
\end{enumerate}


\noindent \textit{Fluxo alternativo}: E-mail já existente na base de dados.

\noindent \textit{Pré-condição}: usuário solicitou atualizar administrador com um endereço de e-mail já existente no banco de dados.

\noindent \textit{Etapas}:

\begin{enumerate}
    \item selecionar o botão ``Atualizar Administrador''; e
    \item sistema retorna a seguinte mensagem: ``E-mail já existe no sistema!''.
\end{enumerate}


%excluir usuário
\vspace{0.7cm}
\leftline{ \textbf{25. Caso de uso}: Excluir usuário.}

\noindent \textit{Pré-condição}: selecionar a opção de excluir o usuário.

\noindent \textit{Fluxo principal}:

\begin{enumerate}
    \item selecionar o ícone de exclusão na linha do registro a ser atualizado;
    \item o sistema exibirá uma mensagem de confirmação de exclusão; e
    \item selecionar o botão ``OK''.
\end{enumerate}

\noindent \textit{Fluxo alternativo}: Cancelamento da exclusão.

\noindent \textit{Pré-condição}: usuário acionou o botão ``Cancelar''.

\noindent \textit{Etapas}:

\begin{enumerate}
    \item selecionar o botão ``Cancelar''.
\end{enumerate}


\noindent \textit{Fluxo alternativo}: Secretário tenta excluir um administrador.

\noindent \textit{Pré-condição}: usuário do tipo ``secretário'' solicitou editar algum administrador do sistema.

\noindent \textit{Etapas}:

\begin{enumerate}
    \item selecionar o botão ``Excluir''; e
    \item sistema retorna a seguinte mensagem: ``Você não tem permissão para realizar essa operação!''.
\end{enumerate}



%recuperar senha
\vspace{0.7cm}
\leftline{ \textbf{26. Caso de uso}: Recuperar senha.}

\noindent \textit{Pré-condição}: selecionar o botão de enviar e-mail.

\noindent \textit{Fluxo principal}:

\begin{enumerate}
    \item selecionar o botão: ``Esqueci a senha'';
    \item digitar o e-mail; e
    \item selecionar o botão ``Enviar''.
\end{enumerate}

\noindent \textit{Fluxo alternativo}: Cancelamento da recuperação.

\noindent \textit{Pré-condição}: usuário acionou o botão ``Voltar''.

\noindent \textit{Etapas}:

\begin{enumerate}
    \item selecionar o botão ``Voltar''.
\end{enumerate}


\noindent \textit{Fluxo alternativo}: E-mail não existente na base de dados.

\noindent \textit{Pré-condição}: usuário solicitou recuperação de senha para um e-mail inexistente no banco de dados do sistema.

\noindent \textit{Etapas}:

\begin{enumerate}
    \item selecionar o botão ``Enviar''; e
    \item sistema retorna a seguinte mensagem: ``E-mail não existe no sistema!''.
\end{enumerate}


\noindent \textit{Fluxo alternativo}: Endereço de e-mail não é válido.

\noindent \textit{Pré-condição}: usuário solicitou recuperação de senha para um e-mail inválido.

\noindent \textit{Etapas}:

\begin{enumerate}
    \item selecionar o botão ``Enviar''; e
    \item sistema retorna a seguinte mensagem: ``E-mail inválido!''.
\end{enumerate}



%gerar carteirinha
\vspace{0.7cm}
\leftline{ \textbf{27. Caso de uso}: Gerar carteirinha.}

\noindent \textit{Pré-condição}: selecionar o ícone de geração de carteirinha para o associado.

\noindent \textit{Fluxo principal}:

\begin{enumerate}
    \item selecionar o ícone de geração de carteirinha para o associado; e
    \item realizar o download ou impressão do documento gerado.
\end{enumerate}

\noindent \textit{Fluxo alternativo}: não se aplica.

\end{document}
