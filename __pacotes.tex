%%%%%%%%%%%%%%%%%%%%%%%%%%%%%%%%%%%%%%%%%%%%%%%%%%%%%%%%%%%%%%%%%%%%%%
%% Esse arquivo inclui os pacotes e configurações utilizados
%% para adequar o modelo.
%%
%%  **** Não é necessário modificar esse arquivo ****
%%
%%%%%%%%%%%%%%%%%%%%%%%%%%%%%%%%%%%%%%%%%%%%%%%%%%%%%%%%%%%%%%%%%%%%%%

\usepackage{scrextend} % para usar o ambiente labeling
\usepackage{array}
\usepackage{scalefnt} %escala de tabelas
\usepackage[brazil]{babel}
\usepackage{mathptmx} %Times new roman para texto e matemática
%\usepackage{times}   %Times new roman para texto
\usepackage{lastpage}
\usepackage{minitoc} 
\usepackage[utf8]{inputenc}
\usepackage{amsmath}
\usepackage{amsfonts}
\usepackage{amssymb}
\usepackage{graphicx}	
\usepackage[]{pdfpages} %permite colocar pdf no texto

\sloppy

%%==================================================================
%% INÍCIO
%% Permite colocar código (Algoritmos) 
%% https://pt.overleaf.com/learn/latex/Code_listing
%%==================================================================
\usepackage{listings} 
\usepackage{xcolor}

%% As linhas abaixo permitem várias personalizações
\definecolor{codegreen}{rgb}{0,0.6,0}
\definecolor{codegray}{rgb}{0.5,0.5,0.5}
\definecolor{codepurple}{rgb}{0.58,0,0.82}
\definecolor{backcolour}{rgb}{0.95,0.95,0.92}
 
\lstdefinestyle{mystyle}{
    backgroundcolor=\color{backcolour},   
    commentstyle=\color{codegreen},
    keywordstyle=\color{magenta},
    numberstyle=\tiny\color{codegray},
    stringstyle=\color{codepurple},
    basicstyle=\ttfamily\footnotesize,
    breakatwhitespace=false,         
    breaklines=true,                 
    captionpos=b,                    
    keepspaces=true,                 
    numbers=left,                    
    numbersep=5pt,                  
    showspaces=false,                
    showstringspaces=false,
    showtabs=false,                  
    tabsize=2
}
 
\lstset{style=mystyle}
%%==================================================================
%% FIM
%% Permite colocar código (Algoritmos) 
%%==================================================================


%%==================================================================
%% INÍCIO
%% Configurações de espaçamento entre títulos dos capítulos e seções
%%==================================================================
% Coloca o espaçamento 1,5 entre títulos de capítulos e o texto. 
% Coloca tanto antes quanto depois. Norma abnt e manual da UFVJM
\setlength\afterchapskip{\lineskip} 
\setlength\beforechapskip{\lineskip}
%coloca o espaçamento 1,5 entre títulos de seções e o texto. 
%Coloca tanto antes quanto depois. Norma abnt e manual da UFVJM
\setlength\aftersecskip{\lineskip}
\setlength\beforesecskip{\lineskip}
%Coloca o espaçamento 1,5 entre títulos de subseções e o texto. 
%Coloca tanto antes quanto depois. Norma abnt e manual da UFVJM
\setlength\aftersubsecskip{\lineskip}
\setlength\beforesubsecskip{\lineskip}
% Coloca o espaçamento 1,5 entre títulos de subsubseções e o texto. 
% Coloca tanto antes quanto depois. Norma abnt e manual da UFVJM
\setlength\aftersubsubsecskip{\lineskip}
\setlength\beforesubsubsecskip{\lineskip}
%%==================================================================
%% FIM 
%% Configurações de espaçamento entre títulos dos capítulos e seções
%%==================================================================

%%==================================================================
%% INÍCIO
%% Euler - pacotes para múltiplas linhas e colunas de tabelas
%%==================================================================
\usepackage{multirow}
\usepackage{multicol}
%%==================================================================
%% FIM 
%% Euler - pacotes para múltiplas linhas e colunas de tabelas
%%==================================================================



%%%marcelo - Caption do tipo do elemento gráfico em negrito.

%\usepackage[labelfont=bf]{caption}

%%% Ex: Figura 1 - Título da figura.
%%% Apenas Figura 1 estará em negrito

%%%marcelo - Todo caption em negrido

\usepackage{caption}
% marcelo - colocar a fonte alinhada a esquerda
%https://tex.stackexchange.com/questions/131532/position-caption-of-centered--left-adjusted-to-figure
%\captionsetup{font={bf,small}, position=below, justification=raggedright, singlelinecheck=false} % negrito e menor
\captionsetup{font={bf,small}, position=below,labelsep=endash} % negrito e menor
%%% tamanhos disponíveis -> scriptsize, footnotesize, small, normalsize, large, Large

%%% Ex: Figura 1 - Título da figura
%%% Figura 1 - Título da figura estará em negrito.
%% Euler: para colocar o travessão: labelsep=endash
% marcelo - Deixa a fonte da figura, quadro, etc, sem negrito quando se utiliza a solução acima.
%\newcommand{\fonteUFVJM}[1]{\legend{\textnormal{Fonte -- #1}}}
\renewcommand{\fonte}[1]{\captionsetup{justification=raggedright, singlelinecheck=false}\legend{\textnormal{Fonte: #1}}}

%\renewcommand{\fonte}[1]{\legend{\textnormal{Fonte: #1}}}


% marcelo - simula aspas duplas no estilo usado pelo JSON (''a'')
\newcommand{\q}[1]{{\textquotesingle\textquotesingle}#1{\textquotesingle\textquotesingle}}
%marcelo - nome menor para backslash
\newcommand{\bks}{\textbackslash}

\usepackage{tabularx} % marcelo - tabelas justificadas
\usepackage{booktabs} % marcelo - grossura da linha em tabelas

%%% marcelo - retira espaços extras entre os elementos de uma lista (itemize e enumarate)
\usepackage{enumitem}
\usepackage{setspace}
\setlist[itemize,enumerate]{nosep} % Caso queira deixar espaçamento duplo entre o parágravo anterior e posterior, mas não entre itens da lista, utilizar noitemsep no lugar de nosep
%%% até aqui

\raggedbottom % marcelo - Resolve o problema do espaçamento entre parágrafos aleatório

\usepackage{subfig} %marcelo - Suporte a subfiguras	

%\usepackage[none]{hyphenat} %marcelo - não permite hifenização


\usepackage{indentfirst}		
\usepackage{color}				
\usepackage{microtype}
%\usepackage[alf,abnt-emphasize=bf]{abntex2cite}

%resolve problemas com as referências
%% Versão justificada das referências
%\usepackage[alf,abnt-etal-cite=3,abnt-etal-list=0,abnt-etal-text=emph,abnt-emphasize=bf]{abntex2cite}

%% Versão alinhada à esquerda das referências
\usepackage[alf,abnt-etal-cite=3,abnt-etal-list=0,abnt-etal-text=emph,abnt-emphasize=bf]{abntex2cite}

\usepackage{ragged2e} %para usar o comando \justify
\usepackage{enumitem} %para mudar as alíneas
\usepackage{textcomp} %fornece caracteres especiais
\setlength{\parindent}{2cm} %Tamanho do parágrafo
%\setlength{\parskip}{1.5cm} 
%\setlrmarginsandblock{3cm}{2cm}


%Para definir o tamanho das fontes das seções, subseções, etc (+++ seria o item (subsection, section, chapter, etc):

\renewcommand{\ABNTEXchapterfontsize}{\normalsize} 
\renewcommand{\ABNTEXsectionfontsize}{\normalsize} 
\renewcommand{\ABNTEXsubsectionfontsize}{\normalsize} 
\renewcommand{\ABNTEXsubsubsectionfontsize}{\normalsize}

%Para deixar as fontes das seções, subseções, etc, em negrito (+++ seria o item (subsection, section, chapter, etc):

\renewcommand{\ABNTEXchapterfont}{\normalfont\normalfont\bfseries}
\renewcommand{\ABNTEXsectionfont}{\normalfont\bfseries}
\renewcommand{\ABNTEXsubsectionfont}{\normalfont\itshape\bfseries}
\renewcommand{\ABNTEXsubsubsectionfont}{\normalfont\itshape}


%Alterar o tipo da fonte no sumário - MARCELO

% Secao secundaria (Section) Caixa baixa, Negrito, tamanho 12
%\renewcommand{\cftsectionfont}{\bfseries} %ponha \rmfamily se quiser serifadas...

% Secao terciaria (Subsection) Caixa baixa e italico no sumário
\renewcommand{\cftsubsectionfont}{\normalfont\itshape\bfseries}

% Secao quaternaria (Subsubsection) Caixa baixa, Negrito, sublinhado, tamanho 12
\renewcommand{\cftsubsubsectionfont}{\normalfont\itshape}

% Seção quinaria (subsubsubsection) Caixa baixa, sem negrito, tamanho 12
%\renewcommand{\cftparagraphfont}{\normalfont}

% MARCELO - Evitar linhas órfãs e viúvas
\widowpenalty=10000
\clubpenalty=10000



%%%%% Euler - Cria comando para citar fontes de figuras e tabelas.

\newcommand{\citefonte}[1]
{\citeauthor{#1}, \citeyear{#1}}


\hypersetup{
     	pagebackref=true,
		pdftitle={\@title}, 
		pdfauthor={\@author},
    	pdfsubject={\imprimirpreambulo},
	    pdfcreator={Nome do Aluno},
		pdfkeywords={abnt}{latex}{abntex}{abntex2}{trabalho acadêmico}, 
		colorlinks=true,       		
    	linkcolor=black,          	
    	citecolor=black,        	
    	filecolor=magenta,      		
		urlcolor=blue,
		bookmarksdepth=4
}

%%==================================================================
%% INÍCIO 
%% João Paulo / Felipe Túlio (PPGEd)
%% Mapas Gráficos ...
%%==================================================================

%% QUADROS
\newcommand{\quadroname}{Quadro}
\newcommand{\listofquadrosname}{Lista de quadros}
\newfloat[chapter]{quadro}{loq}{\quadroname}
\newlistof{listofquadros}{loq}{\listofquadrosname}
\newlistentry{quadro}{loq}{0}
% configurações para atender às regras da ABNT
\setfloatadjustment{quadro}{\centering}
\counterwithout{quadro}{chapter}
\renewcommand{\cftquadroname}{\quadroname\space} 
\renewcommand*{\cftquadroaftersnum}{\hfill--\hfill}
% Configuração de posicionamento padrão:
\setfloatlocations{quadro}{hbtp}

%% Mapas
\newcommand{\mapaname}{Mapa}
\newcommand{\listofmapasname}{}
\newfloat[chapter]{mapa}{lom}{\mapaname}
\newlistof{listofmapas}{lom}{\listofmapasname}
\newlistentry{mapa}{lom}{0}
\setfloatadjustment{mapa}{\centering}
\counterwithout{mapa}{chapter}
\renewcommand{\cftmapaname}{\mapaname\space} 
\renewcommand*{\cftmapaaftersnum}{\hfill--\hfill}
\setfloatlocations{mapa}{hbtp}

%% Graficos
\newcommand{\graficoname}{Gráfico}
\newcommand{\listofgraficosname}{}
\newfloat[chapter]{grafico}{logrf}{\graficoname}
\newlistof{listofgraficos}{logrf}{\listofgraficosname}
\newlistentry{grafico}{logrf}{0}
\setfloatadjustment{grafico}{\centering}
\counterwithout{grafico}{chapter}
\renewcommand{\cftgraficoname}{\graficoname\space} 
\renewcommand*{\cftgraficoaftersnum}{\hfill--\hfill}
\setfloatlocations{grafico}{hbtp}

%% Algoritmo
\newcommand{\algoritmoname}{Algoritmo}
\newcommand{\listofalgoritmosname}{Lista de Ilustrações} % Gambiarra 
\newfloat[chapter]{algoritmo}{loalg}{\algoritmoname}
\newlistof{listofalgoritmos}{loalg}{\listofalgoritmosname}
\newlistentry{algoritmo}{loalg}{0}
\setfloatadjustment{algoritmo}{\centering}
\counterwithout{algoritmo}{chapter}
\renewcommand{\cftalgoritmoname}{\algoritmoname\space} 
\renewcommand*{\cftalgoritmoaftersnum}{\hfill--\hfill}
\setfloatlocations{algoritmo}{hbtp}

% Gambiarra 
\addto\captionsbrazil{%
  \renewcommand{\listfigurename}%
    {}%
}

%%==================================================================
%% FIM 
%% João Paulo / Felipe (PPGEd)
%% Mapas Gráficos ...
%%==================================================================




%%==================================================================
%% INÍCIO 
%% Alexandre - Comandos adicionais 
%%==================================================================

% Comando para definir curso/programa
\providecommand{\imprimircurso}{}
\newcommand{\curso}[1]{\renewcommand{\imprimircurso}{#1}}

% Comando para Titulação do Orientador
\providecommand{\imprimirTitulacaoOrientador}{}
\newcommand{\TitulacaoOrientador}[1]{\renewcommand{\imprimirTitulacaoOrientador}{#1}}

% Comando para Titulação do Orientador
\providecommand{\imprimirDepartamentoOrientador}{}
\newcommand{\DepartamentoOrientador}[1]{\renewcommand{\imprimirDepartamentoOrientador}{#1}}

% Comando para Titulação do Coorientador
\providecommand{\imprimirTitulacaoCoorientador}{}
\newcommand{\TitulacaoCoorientador}[1]{\renewcommand{\imprimirTitulacaoCoorientador}{#1}}

% Comando para Titulação do Orientador
\providecommand{\imprimirDepartamentoCoorientador}{}
\newcommand{\DepartamentoCoorientador}[1]{\renewcommand{\imprimirDepartamentoCoorientador}{#1}}

%Membos da banca 
\providecommand{\imprimirMembroA}{}
\newcommand{\membroA}[1]{\renewcommand{\imprimirMembroA}{#1}}
\providecommand{\imprimirMembroB}{}
\newcommand{\membroB}[1]{\renewcommand{\imprimirMembroB}{#1}}
\providecommand{\imprimirMembroC}{}
\newcommand{\membroC}[1]{\renewcommand{\imprimirMembroC}{#1}}
\providecommand{\imprimirMembroD}{}
\newcommand{\membroD}[1]{\renewcommand{\imprimirMembroD}{#1}}
\providecommand{\imprimirTitulacaoMembroA}{}
\newcommand{\TitulacaoMembroA}[1]{\renewcommand{\imprimirTitulacaoMembroA}{#1}}
\providecommand{\imprimirTitulacaoMembroB}{}
\newcommand{\TitulacaoMembroB}[1]{\renewcommand{\imprimirTitulacaoMembroB}{#1}}
\providecommand{\imprimirTitulacaoMembroC}{}
\newcommand{\TitulacaoMembroC}[1]{\renewcommand{\imprimirTitulacaoMembroC}{#1}}
\providecommand{\imprimirTitulacaoMembroD}{}
\newcommand{\TitulacaoMembroD}[1]{\renewcommand{\imprimirTitulacaoMembroD}{#1}}
\providecommand{\imprimirDepartamentoMembroA}{}
\newcommand{\DepartamentoMembroA}[1]{\renewcommand{\imprimirDepartamentoMembroA}{#1}}
\providecommand{\imprimirDepartamentoMembroB}{}
\newcommand{\DepartamentoMembroB}[1]{\renewcommand{\imprimirDepartamentoMembroB}{#1}}
\providecommand{\imprimirDepartamentoMembroC}{}
\newcommand{\DepartamentoMembroC}[1]{\renewcommand{\imprimirDepartamentoMembroC}{#1}}
\providecommand{\imprimirDepartamentoMembroD}{}
\newcommand{\DepartamentoMembroD}[1]{\renewcommand{\imprimirDepartamentoMembroD}{#1}}
%%==================================================================
%% FIM
%% Alexandre - Comandos adicionais
%%==================================================================

%%==================================================================
%% INÍCIO CAPA
%% Define a capa do trabalho
%%==================================================================
\renewcommand{\imprimircapa}{% 
\begin{capa}% 
    \center 
    {\bfseries \imprimirinstituicao}\\
    {\bfseries \imprimircurso}\\
    {\bfseries \imprimirautor} 
    %\vspace*{1cm} 
    \vfill 
    \begin{center} 
    \bfseries \imprimirtitulo 
    \end{center} 
    \vfill 
    \bfseries \imprimirlocal \\
    \bfseries \imprimirdata 
    \vspace*{1cm} 
\end{capa} 
}%
%%==================================================================
%% FIM CAPA
%% Define a capa do trabalho
%%==================================================================

%%==================================================================
%% INÍCIO FOLHA DE ROSTO
%% Define a folha de rosto
%%==================================================================
\makeatletter
\renewcommand{\folhaderostocontent}{ 
\begin{center} 
    {\bfseries\imprimirautor} 
    \vspace*{\fill}
    \vspace*{\fill} 
    \begin{center} \bfseries\imprimirtitulo \end{center} 
    \vspace*{\fill} 
    \abntex@ifnotempty{\imprimirpreambulo}{% 
    \hspace{.45\textwidth} 
    \begin{minipage}{.5\textwidth} 
        \begin{SingleSpacing}
        \imprimirpreambulo
        \end{SingleSpacing}
        \begin{SingleSpacing}
            {\imprimirorientadorRotulo~\imprimirorientador}\\ 
            \abntex@ifnotempty{\imprimircoorientador}{% 
            {\imprimircoorientadorRotulo~\imprimircoorientador}% 
            }% 
        \end{SingleSpacing}
    \end{minipage}%
    \vspace*{\fill} }% 
    %{\abntex@ifnotempty{\imprimirinstituicao}{\imprimirinstituicao \vspace*{\fill}}} \\
    \vspace*{\fill} \\
    {\bfseries\imprimirlocal} 
    \par 
    {\bfseries\imprimirdata} 
    \vspace*{1cm} 
\end{center} 
} 
\makeatother 
%%==================================================================
%% FIM FOLHA DE ROSTO
%% Define a folha de rosto
%%==================================================================


\makeindex

