\chapter{Considerações Finais}

O trabalho consistiu na criação de um sistema robusto subdividido em um portal, gerenciado por colaboradores mas acessível a todo tipo de público, e um sistema gerencial, utilizado por secretários e o presidente da associação Kuruatuba. 

Durante a documentação do projeto, foram abordados os seguintes tópicos: apresentação da Kuruatuba e dos objetivos do trabalho; divulgação e estudo de pesquisas referenciais relacionadas a ferramentas, arquiteturas e conceitos aplicados neste trabalho; e explanação sobre como foram realizadas as atividades, desde a definição e justificativa de escolha para cada tecnologia até o detalhamento do passo a passo da implementação do software.

Por fim, vale destacar que todos os objetivos, geral e específicos, listados na Seção \ref{sec:telas} foram atingidos. Além deles, o produto final agregou as funcionalidades de se ter dados gerais e específicos sobre os visitantes do portal, concedendo à organização a oportunidade de acessar e mensurar o alcance de suas informações públicas em todo mundo, e de se permitir uma comunicação direta entre o visitante e a associação por meio de formulário eletrônico, possibilitando um maior engajamento entre as partes pela praticidade na comunicação. 

%falar se atingiu ou não os objetivos gerais e específicos

\hspace{2.5cm}
\section{Trabalhos futuros}
\hspace{2.5cm}

Como trabalhos futuros, pode-se elencar:

\begin{itemize}
 \item criar um manual de utilização e executar um treinamento aos futuros usuários do portal e do sistema de associados;
 \item instalar um certificado digital (SSL, do inglês \textit{Secure Sockets Layer}) \footnote{Tecnologia que provê a utilização de um protocolo como o HTTPS e o TLS) com o intuito de criptografar dados transitados entre formulários e o servidor. Normalmente utilizada em páginas que contêm formulário de cadastro ou de login para prevenir interceptação da senha descriptografada enviada pelo cliente. Saiba mais em: \citeonline{sslpage}.} ao menos nas telas de login de ambos sistemas;
 \item executar testes de software afim de se obter \textit{feedbacks} sobre integridade, disponibilidade e acessibilidade do sistema como um todo.
\end{itemize}


%falar sobre o treinamento que os secretários da associação terão para manipular o sistema
%falar sobre a possibilidade de aplicar testes de software a fim de obter mais feedbacks sobre integridade, disponibilidade e acessibilidade do sistema
%falar sobre a instalação do certificado SSL no site
