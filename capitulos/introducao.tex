\newpage
\chapter{Introdução}
\hspace{2.5cm}

\section{Apresentação}
Atualmente os sistemas web fazem parte do cotidiano das pessoas que estão cada vez mais conectadas a um mundo virtual, também denominado internet, fundamentalmente sustentado pela comunicação com ou entre seus usuários.
Deste modo, qualquer conteúdo existente nele tem um significado que deve ser interpretado por um ser humano, ou seja, havendo a transmissão de informação e pressupondo um processo de comunicação \cite{de2012rede}. Para manter a interação entre as pessoas, existem as aplicações web, por exemplo as redes sociais, manipuladas diariamente por grande parte dos usuários interativos da internet, também conhecidos como internautas. 

Existem diversos tipos de aplicações para a internet, \citeonline{gonccalves2005proposta} e \citeonline{de2003educaccao} citam como exemplares as páginas para a web, aplicações \textit{E-business}, aplicações de comércio eletrônico, a educação on-line e a \textit{E-Learning} (aprendizagem eletrônica). Antes de dar prosseguimento ao conteúdo, é interessante explicar a diferença entre os conceitos de \textit{website} e sistemas web, pois são termos que muitos, inclusive programadores, consideram ter o mesmo significado. 

Um \textit{website} é basicamente um conjunto de páginas com caráter meramente expositivo, não sendo possível ocorrer consultas em bancos de dados. As páginas que contém conteúdo como informações para contato, missão e valores, e história são exemplos. Já nas aplicações ou sistemas web, ocorre a interação entre o software e o usuário, seja por formulários de cadastro, consultas a informações contidas no bancos de dados ou outros recursos específicos de cada tipo de aplicação. Segundo \citeonline{garrett2005ajax}, o modelo clássico para aplicações web funciona da seguinte maneira: ações do usuário pela interface do sistema acionam uma solicitação para o servidor, que processa dados e informações retornando-os para o cliente (usuário) através de uma página HTML
(\textit{Hypertext Markup Language}).

Em algumas instituições é comum perceber que alguns ou vários processos administrativos ainda são feitos manualmente, sem uso de sistemas informatizados que auxiliem em seu desenvolvimento. Esse é um fato preocupante, porque tarefas básicas e rotineiras, que poderiam ser realizadas de forma mais rápida e eficaz por sistemas informatizados, acabam consumindo mais tempo para serem executadas e ainda tenderão a erros humanos que possam estar ocorrendo durante ou após o processo \cite{othman2009development}. 

\begin{citacao}
Indústrias como fabricação, viagens e hospitalidade, bancos, educação e governo estão habilitados na web para melhorar e aprimorar suas operações. O comércio eletrônico expandiu-se rapidamente, atravessando fronteiras nacionais. Até os sistemas tradicionais de informações e bancos de dados herdados migraram para a Web \cite[p. 1]{ginige2001web}.
\end{citacao}

Diante do cenário tecnológico global e da influência dos sistemas integrados a internet na vida das pessoas, manifestou-se a oportunidade de criação de um software, que agregado às características típicas de aplicações web, conseguisse expandir seus benefícios para o dia-a-dia dos membros e desfrutadores das ações desempenhadas pela associação Kuruatuba.

\hspace{2.5cm}
\section{Sobre a Kuruatuba}
\hspace{2.5cm}

Em seu blog, \citeonline{Kuruatuba2011} discorre sobre seu histórico. Segundo o texto, a história da Kuruatuba está atrelada à utilização das areias da praia do Copo Sujo, localizada no município de Janaúba-MG, para prática de esportes juntamente com o combate às diversas ameaças ao meio ambiente ocorridas no ano de 1988. A partir de 1989, ano marcado pela realização do evento ``Carnaval 40º Graus'', que teve ampla repercussão e adesão pela comunidade local com o objetivo de valorização do ambiente, a praia foi zelada por um grupo de atletas de vôlei e futebol de areia, que em 1998 se uniram e fundaram a Associação de Futebol de Praia do Copo Sujo de Janaúba (AFPJ).

A AFPJ foi parceira de diversos órgãos e instituições durante a conservação da praia do Copo Sujo. Trabalhos foram desenvolvidos com apoio da Secretaria de Meio Ambiente da Prefeitura de Janaúba e Ruralminas, IEF, CODEMA, Poder Judiciário (Albergados), escolas, igrejas e diversos segmentos da sociedade. Em 2003, com uma evidência mais ambientalista, mesmo ainda empenhando-se na atividades de preservação do Rio Gorutuba, a AFPJ passou a ser intitulada Kuruatuba, que significa ``sapo grande'', homenagem ao rio.  

Em novembro de 2003 foi aprovado o estatuto de regimento da Kuruatuba - Associação dos Protetores da Bacia Hidrográfica do Rio Gorutuba De Janaúba-mg contendo 45 artigos que apresentam valores, normas, objetivos e diversos outros esclarecimentos sobre a mesma. Ainda segundo \citeonline{Kuruatuba2011}, o objetivo da associação ``é promover o esporte, lazer, cultura e preservação e conservação da Bacia do Rio Gorutuba, sendo sua área de atuação compreendida da nascente à foz, incluindo os seus afluentes.''

\hspace{2.5cm}
\section{Justificativa}
\hspace{2.5cm}

A ideia para o projeto partiu da própria associação, presidida pelo professor Erinaldo Barbosa da Silva. Na época, a instituição passava por dificuldades para ampliar o número de apoiadores e associados. O blog da Kuruatuba era o único local propriamente da associação fora das redes sociais, porém, a informalidade transmitida por um blog vem a gerar uma desconfiança por parte do visitante por este não ter noção exata do tamanho e da importância da instituição para a sociedade. O baixo grau de personalização do \textit{layout} de um blog também é um ponto negativo, visto que em um site ou portal é possível organizar todos os elementos da maneira que a organização desejar, embora seja necessário conhecimento sobre linguagens de programação. 

Blogs, em sua essência, possuem um caráter informal, conforme \citeonline{centeno2017pampatur} afirma. Segundo a autora, blogs são destinados para divulgação de opiniões de pessoas individuais ou de grupos sobre assuntos específicos, enquanto que sites são focados para empresas e apresentam conteúdos formais e pouco dinâmicos, estando ou não exibidos para o visitante de forma cronológica, diferentemente de blogs. 

Um blog tem ainda limitações relacionadas a segurança por não ter níveis de acesso por usuário, com o intuito de determinar quais conteúdos poderão ser acessados ou editados por cada contribuidor. Blogs também possuem limitações para uma empresa ou instituição para gestão de colaboradores, remunerados ou não, e também de publicação de balanços patrimoniais, processos seletivos e outros documentos importantes. 

Em levantamento inicial de dados e informações para construção do projeto, disponível no Apêndice \ref{ch:pesquisa}, no qual foram coletadas respostas de apoiadores e associados via formulário on-line, obteve-se que apenas 7,1\% dos participantes recebem notícias referentes à associação via site ou blog, e 56,3\% gostariam de receber tais notícias por esses veículos. Com isso, renova-se a ideia de criar uma aplicação na internet onde notícias e eventos possam ser cadastrados e publicados por meio eficiente para disseminação.

Com a intenção de reduzir as dificuldades mencionadas e criar um ambiente destinado a suprir as principais necessidades de associados, membros da gestão institucional, colaboradores e demais públicos, o projeto fora aprovado, dando início a um período de significante aprendizado e dedicação.

%Realizar uma entrevista com Erinaldo para saber sobre como é feito o processo de gerenciamento de associados, dificuldades encontradas para realizar a divulgação de informações e eventos, conseguir mais apoiadores e verbas... 
\hspace{2.5cm}
\section{Objetivos}
\label{sec:telas}
\hspace{2.5cm}

Nesta seção, serão apresentados os objetivos gerais e específicos que regem todo o trabalho. 

O objetivo geral é a construção de um sistema de informação capaz de auxiliar na gestão das informações pertencentes à associação Kuruatuba, também possibilitando a divulgação de informes e a propagação de seus ideais de maneira mais rápida e abrangente.

Considerando-se, agora, os objetivos específicos escolhidos, encontram-se os listados abaixo:  
\begin{itemize}
 \item oferecer aos seus usuários e visitantes as principais funcionalidades para a manutenção, segurança e disponibilidade das informações;
 
 \item promover a divulgação de notas, informes, eventos organizados e demais conteúdos desejáveis de autoria atribuída à associação Kuruatuba de maneira organizada e intuitiva para o usuário;
  
 \item possibilitar o gerenciamento de pessoas associadas à organização, por meio de sistema com interface agradável e opção de emissão de documento comprobatório de associativismo para os cadastrados; e
 
 \item tornar o sistema responsivo, sendo possível acessá-lo em diferentes dispositivos físicos com qualidade equivalente à de \textit{desktops}.  
\end{itemize}

%o fato de gostarem de redes sociais, alimentam a necessidade de incluir links para elas no próprio site


