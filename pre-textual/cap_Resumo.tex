\chapter*{Resumo}

\vspace{0.4cm}

\noindent O presente trabalho teve por finalidade empregar conteúdos estudados durante o curso de Sistemas de Informação e tecnologias para desenvolvimento de aplicações web na construção de um sistema eletrônico para a Associação dos Protetores da Bacia Hidrográfica do Rio Gorutuba (Kuruatuba), localizada no município de Janaúba, região norte do estado de Minas Gerais. 
A Kuruatuba é uma sociedade civil sem fins lucrativos que desempenha importantes funções não só na preservação da bacia do rio Gorutuba, como também na realização de atividades comunitárias com o objetivo de promover bem estar social para a comunidade em geral. 
Neste trabalho foi apresentada a criação de um sistema informatizado pelo qual será possível: agregar mais visibilidade e notoriedade à Kuruatuba, permitir que o público em geral tenha acesso a informações públicas e acompanhe suas ações realizadas e possibilitar à associação gerenciar intuitivamente seus conteúdos digitais e colaboradores.  
Dentre todas as ferramentas manipuladas e metodologias adotadas no desenvolvimento do software, destacam-se as seguintes: \textit{Docker, Plone}, PHP, MySQL e \textit{Scrum}.
Como resultado obtido vale à pena mencionar o próprio sistema, composto pelo portal e pelo sistema gerencial de associados, que não só respeita todas as exigências do cliente obtidas via processo de elicidação de requisitos, como também pode ser utilizado para coletar diversos dados públicos sobre os visitantes através do \textit{Google Analytics}, estando apto a ser plenamente manuseado após instalação de certificado SSL e elaboração de um tutorial com instruções sobre sua utilização.
No decorrer do trabalho foram apresentados seus reais objetivos, uma abordagem de natureza bibliográfica sobre ferramentas e conceitos utilizados no projeto, as etapas de implementação do sistema e, ao final, as metas alcançadas e as propostas para uma possível continuação do projeto.

\begin{labeling}{\textbf{Palavras-chave:}}
\item[\textbf{Palavras-chave:}] 
Engenharia de Software.
\textit{Docker}.
\textit{Plone}.
PHP.
MySQL.
\textit{Scrum}.
\end{labeling}

%%%%%%%%%%%%%%%%%%%%%%%%%%%%%%%%%%%%%%%%%%%%%%%%%%%%%%%%%%%%%%%%%%%%%%
%%% Comente o texto abaixo 
% \textcolor{red}{(as palavras-chave são separadas por ponto) (deve-se dar preferência ao uso de palavras-chave cadastradas em \url{http://acervo.bn.br/sophia_web/index.html}.)}
