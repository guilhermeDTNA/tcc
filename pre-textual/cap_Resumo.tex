\chapter*{Resumo}

\vspace{0.4cm}

\noindent O presente trabalho teve por finalidade empregar conteúdos estudados durante o curso de Sistemas de Informação e tecnologias para desenvolvimento de aplicações web na construção de um sistema eletrônico para a Associação dos Protetores da Bacia Hidrográfica do Rio Gorutuba (Kuruatuba), localizada no norte de Minas Gerais. A Kuruatuba é uma sociedade civil sem fins lucrativos que desempenha importantes funções não só na preservação da Bacia do Rio Gorutuba, como também na realização de atividades comunitárias com o objetivo de promover bem estar social para a comunidade em geral. O intuito deste trabalho é agregar mais visibilidade e notoriedade à associação e permitir a mobilização da comunidade local para as ações realizadas pela Kuruatuba e sua importância.   
Dentre todas as ferramentas manipuladas e metodologias adotadas no desenvolvimento do software, destacam-se as seguintes: \textit{Docker, Plone}, PHP, MySQL e \textit{Scrum}.  
No decorrer do trabalho foram apresentados e justificados os seus reais objetivos, as etapas de implementação do sistema e, ao final, as metas alcançadas e as propostas para uma possível continuação do projeto.

\begin{labeling}{\textbf{Palavras-chave:}}
\item[\textbf{Palavras-chave:}] 
Engenharia de Software.
\textit{Docker}.
\textit{Plone}.
PHP.
MySQL.
\textit{Scrum}.
\end{labeling}

%%%%%%%%%%%%%%%%%%%%%%%%%%%%%%%%%%%%%%%%%%%%%%%%%%%%%%%%%%%%%%%%%%%%%%
%%% Comente o texto abaixo 
% \textcolor{red}{(as palavras-chave são separadas por ponto) (deve-se dar preferência ao uso de palavras-chave cadastradas em \url{http://acervo.bn.br/sophia_web/index.html}.)}
