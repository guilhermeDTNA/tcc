\chapter*{Resumo}

\vspace{0.4cm}

\noindent O trabalho abordado neste documento teve por finalidade empregar conteúdos estudados durante o curso de Sistemas de Informação e tecnologias recentes de desenvolvimento de aplicações web utilizadas em órgãos nacionais e internacionais, na construção de um sítio eletrônico para a associação de protetores da bacia hidrográfica do Rio Gorutuba de Janaúba/MG (Kuruatuba), localizada no norte de Minas Gerais. A Kuruatuba é uma sociedade civil sem fins lucrativos que desempenha importantes funções não só na preservação da Bacia do Rio Gorutuba, como também na realização de atividades comunitárias para promoção de bem estar social para a comunidade em geral. O trabalho objetiva-se agregar mais visibilidade e notoriedade à associação e permitir a mobilização da comunidade local para as ações realizadas pela Kuruatuba e sua importância.   
Entre as ferramentas usadas no desenvolvimento do software, podemos destacar as seguintes: \textit{Docker, Plone, Google Analytics, Apache JMeter} e \textit{Git}.  
No decorrer do trabalho foram apresentados e justificados os seus reais objetivos, e, ao final, as metas alcançadas e propostas para uma possível continuação do projeto, após a obtenção dos resultados .

\begin{labeling}{\textbf{Palavras-chave:}}
\item[\textbf{Palavras-chave:}] 
Engenharia de Software.
Engenharia Web.
\textit{Docker}.
\textit{Plone}.
\textit{Scrum}.
\end{labeling}

%%%%%%%%%%%%%%%%%%%%%%%%%%%%%%%%%%%%%%%%%%%%%%%%%%%%%%%%%%%%%%%%%%%%%%
%%% Comente o texto abaixo 
% \textcolor{red}{(as palavras-chave são separadas por ponto) (deve-se dar preferência ao uso de palavras-chave cadastradas em \url{http://acervo.bn.br/sophia_web/index.html}.)}
