\begin{resumo}
O trabalho abordado neste documento teve por finalidade empregar conte�dos estudados durante o curso sobre Engenharia de Software e tecnologias recentes de desenvolvimento de aplica��es web, utilizadas em �rg�os nacionais e internacionais, na constru��o de um s�tio eletr�nico para a associa��o Kuruatuba, localizada no norte de Minas Gerais. A associa��o de protetores da bacia hidrogr�fica do Rio Gorutuba de Jana�ba/MG (KURUATUBA) � uma sociedade civil sem fins lucrativos que desempenha importantes fun��es n�o s� na preserva��o da Bacia do Rio Gorutuba, como tamb�m na realiza��o de atividades comunit�rias para promo��o de bem estar social para a comunidade em geral.
Entre as ferramentas usadas no desenvolvimento do software, podemos destacar as seguintes: \textit{Docker, Plone, Google Analytics, Apache JMeter} e \textit{Git}.  
No decorrer do trabalho foram apresentados e justificados os seus reais objetivos, e, ao final, as metas alcan�adas e propostas para uma poss�vel continua��o do projeto, ap�s a obten��o dos resultados .

\noindent \textbf{Palavras-chave:} Engenharia de Software, Engenharia Web, \textit{Docker, Plone, Scrum}.
\end{resumo}
