\documentclass{beamer}
% \usepackage[utf8]{inputenc}
\usepackage{default}
\mode<presentation>{
    %\usepackage{beamerthemeshadow}
    %\usepackage{beamerthemeshadow}
    %\usepackage{beamerthemePaloAlto}
    %\usepackage{beamerthemeSzeged}
    %\usepackage{beamerthemeAntibes}
    %\usepackage{beamerthemeBergen}
    %\usepackage{beamerthemeBerkeley}
    %\usepackage{beamerthemeBerlin}
    %\usepackage{beamerthemeBoadilla}  
    %\usepackage{beamerthemeboxes}  
    %\usepackage{beamerthemeCambridgeUS}
    %\usepackage{beamerthemeCopenhagen} 
    %\usepackage{beamerthemeDarmstadt}
    %\usepackage{beamerthemeDresden}
    %\usepackage{beamerthemeFrankfurt}
    %\usepackage{beamerthemeGoettingen}
    %\usepackage{beamerthemeHannover}
    \usepackage{beamerthemeIlmenau}
    %\usepackage{beamerthemeJuanLesPins} 
    %\usepackage{beamerthemeLuebeck} 
    \usepackage{pgf,pgfarrows,pgfnodes,pgfautomata,pgfheaps,pgfshade}
    \usepackage{clrscode3e}
    \beamertemplatetransparentcovereddynamic
    \beamertemplateballitem
    %\beamertemplatefootpagenumber
}
\mode<handout>{
    \usepackage[bar]{beamerthemetree}
    % Colocando um fundo cinza quando for gerar transparências para serem impressas
    % mais de uma transparência por página
    \beamertemplatesolidbackgroundcolor{black!5}
}
% \usepackage[ruled,chapter]{algorithm}
\usepackage{algorithm}	
\usepackage{algorithmic}

\usepackage{comment}
\usepackage{amsmath,amssymb}
\usepackage[brazil]{varioref}
\usepackage[english,brazil]{babel}
\usepackage[utf8]{inputenc}
\usepackage{graphicx}
\usepackage[alf]{abntex2cite}
\usepackage{abntex2abrev}
% \usepackage[round,sort,authoryear,nonamebreak]{natbib}
% \usepackage[authoryear]{natbib}
%\usepackage{minitoc-hyperref}
%\usepackage{listings}
%\usepackage{listings}
%\usepackage{colortbl}
%\usepackage{pslatex}
\beamertemplatetransparentcovereddynamic

\usepackage{tikz}
\usetikzlibrary{arrows}
\tikzstyle{block}=[draw opacity=0.7,line width=1.4cm]

\newcommand*\oldmacro{}%
\let\oldmacro\insertshorttitle%
\renewcommand*\insertshorttitle{%
  \oldmacro\hfill%
  \insertframenumber\,/\,\inserttotalframenumber}

\title[Trabalho de Conclusão de Curso]{Trabalho de Conclusão de Curso}
\subtitle{Desenvolvimento de um sistema web para a Associação dos Protetores da Bacia hidrográfica do Rio Gorutuba ``Kuruatuba''}

%\titlegraphic{}

\author[Guilherme Rocha Leite]{Guilherme Rocha Leite}
  \institute[UFVJM]{Universidade Federal dos Vales do Jequitinhonha e Mucuri \newline
  Bacharelado em Sistemas de Informação \newline
%   \inst{Departamento de Computação
	  
     Orientador: Prof.ª Erinaldo Barbosa da Silva\\
     Co-orientador: Thales Francisco Mota Carvalho\\
     $~$\\
%      Disciplina: Nome da Disciplina\\
}
\logo{\includegraphics[scale=0.1]{logo-ufvjm.jpg}}
% Se comentar a linha abaixo, irá aparecer a data quando foi compilada a apresentação
%\date{\textcolor{red}{III Encontro Mineiro de Equações Diferenciais, 2009}}
%\pgfdeclareimage[height=0.4cm]{das}{figs/logodas}
%\pgfdeclareimage[height=1cm]{logo}{logos}
% pode-se colocar o LOGO assim
%\logo{\pgfuseimage{logo}}
% ou...
%\logo{\vbox{\hbox to 3cm{\hfil\pgfuseimage{logo}}}}

\begin{document}
%\xdefinecolor{MyGreen}{rgb}{0,0.6,0}
%\beamerboxesdeclarecolorscheme{MyGreen}{MyGreen}{MyGreen!20!averagebackgroundcolor}
\beamerboxesdeclarecolorscheme{formula}{white}{blue!250!averagebackgroundcolor}
\frame{\titlepage}

%\frame{\titlepage}
\part{Presentation}
% \section{Sumário}

\frame{
\frametitle{Sumário}
\tableofcontents
}

\AtBeginSection[]{
  \frame<handout:0>{
    %\frametitle{Sumário}
    \tableofcontents[current,currentsection]
  }
}

%===================================Slide=================================================

\section{Introdução}

\begin{frame}
    \frametitle{Sobre a Kuruatuba}
    \begin{itemize}
        \item \textbf{Fundação:} 1989 - Associação de Futebol de Praia do Copo Sujo de Janaúba;
        \item \textbf{Objetivo:} promoção de esporte, lazer, cultura e preservação e conservação da Bacia do Rio Gorutuba;
        \item \textbf{Parcerias:} Secretaria de Meio Ambiente da Prefeitura de Janaúba e Ruralminas, IEF, CODEMA, Poder Judiciário (Albergados), escolas, igrejas e outros segmentos;
        \item \textbf{Criação do estatuto:} 2003.
    \end{itemize}

\end{frame}


%pag 20
%justificativa 
\begin{frame}
    \frametitle{Motivação}
    \textbf{Principais dificuldades:}
    \begin{itemize}
     \item Impulsionar publicações e atrair apoiadores;
     \item Gerenciar pessoas e arquivos digitais relacionadas à associação.
    \end{itemize}

    
    \textbf{Problemas encontrados:}
    \begin{itemize}
        \item Informalidade em utilização de blog: opiniões pessoais para assuntos específicos \cite{centeno2017pampatur};
        \item Baixos controle de usuários e personalização visual de um blog; %não tem tanto nível hierárquico entre administradores
        \item Baixa capacidade de atingir o público alvo em publicações; %apenas 7,1% recebem notícias via site ou blog e 56,3% gostariam
        \item Dificuldade em manter membros e associados registrados e atualizados.
    \end{itemize}

\end{frame}


%pag 21
\begin{frame}
    \frametitle{Objetivos}
    \textbf{Objetivo geral:} construção de um sistema web para auxiliar na gestão das informações e propagação de conteúdo de autoria da Kuruatuba.
    
    \textbf{Objetivos específicos:}
    \begin{itemize}
        \item Oferecer manutenção, segurança e disponibilidade das informações;
        \item Promover divulgação de vários tipos de conteúdo de maneira organizada;
        \item Possibilitar o gerenciamento de pessoas e administradores vinculados à Kuruatuba;
        \item Aprofundar estudos sobre engenharia e desenvolvimento de software e segurança de dados.
    \end{itemize}
\end{frame}


%pags 24 e 28
%Pra que CMS e Metodologias de desenvolvimento
\section{Desenvolvimento}
\begin{frame}
    \frametitle{Escolha das Ferramentas}
    \begin{itemize}
        \item pra 
    \end{itemize}
\end{frame}


%pag 49
\begin{frame}
    \frametitle{Escolha do Gerenciador de Conteúdos}
    \begin{itemize}
        \item Surgiu...
    \end{itemize}
\end{frame} 


%pag 41
\begin{frame}
    \frametitle{Escolha da Metodologia de Desenvolvimento}
    \begin{itemize}
        \item \textbf{Docker}
        \item \textbf{Git}
    \end{itemize}
\end{frame}


%pag 50
\begin{frame}
    \frametitle{Escolha da Metodologia de Desenvolvimento}
    \begin{itemize}
        \item \textbf{Docker}
        \item \textbf{Git}
    \end{itemize}
\end{frame}


%pag 40
\begin{frame}
    \frametitle{Outras Ferramentas Utilizadas}
    \begin{itemize}
        \item 
    \end{itemize}
\end{frame}


%pag 54
\begin{frame}
    \frametitle{Coleta de Requisitos}
    \textbf{Recursos utilizados:}
    \begin{itemize}
        \item \textbf{Reuniões e questionários}
        \item \textbf{Histórias de usuário}
        \item \textbf{Diagramas de casos de uso}
        \item \textbf{Fluxos de eventos}
    \end{itemize}
\end{frame}


%pag 56
\begin{frame}
    \frametitle{Diagramas de Casos de Uso}
    \begin{itemize}
        \item \textbf{Docker}
        \item \textbf{Git}
    \end{itemize}
\end{frame}


%pag 74
\begin{frame}
    \frametitle{\textit{Product Backlog}}
    \begin{itemize}
        \item \textbf{Docker}
        \item \textbf{Git}
    \end{itemize}
\end{frame}


%pag 76
\begin{frame}
    \frametitle{Estrutura de \textit{containers} do sistema}
    \begin{itemize}
        \item \textbf{Docker}
        \item \textbf{Git}
    \end{itemize}
\end{frame}


%pag 79
\section{Resultados}
\begin{frame}
    \frametitle{Apresentação das telas}
    \begin{itemize}
        \item \textbf{Docker}
        \item \textbf{Git}
    \end{itemize}
\end{frame}


%pag 84
\section{Discussão}
\begin{frame}
    \frametitle{Discussão}
    \begin{itemize}
        \item \textbf{Docker}
        \item \textbf{Git}
    \end{itemize}
\end{frame}


%pag 87
\section{Trabalhos futuros}
\begin{frame}
    \frametitle{Trabalhos futuros}
    \begin{itemize}
        \item \textbf{Docker}
        \item \textbf{Git}
    \end{itemize}

    
\end{frame}

    

\section{Referências Bibliográficas}

\begin{frame}
    \frametitle{Referências Bibliográficas}
    \scriptsize{  
        \bibliographystyle{abnt-alf}
        \bibliography{referencias}
    }
  
\end{frame}

\end{document}
