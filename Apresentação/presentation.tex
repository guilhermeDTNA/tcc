\documentclass[xcolor=table]{beamer}
% \usepackage[utf8]{inputenc}
\usepackage{default}
\mode<presentation>{
    %\usepackage{beamerthemeshadow}
    %\usepackage{beamerthemeshadow}
    %\usepackage{beamerthemePaloAlto}
    %\usepackage{beamerthemeSzeged}
    %\usepackage{beamerthemeAntibes}
    %\usepackage{beamerthemeBergen}
    %\usepackage{beamerthemeBerkeley}
    %\usepackage{beamerthemeBerlin}
    %\usepackage{beamerthemeBoadilla}  
    %\usepackage{beamerthemeboxes}  
    %\usepackage{beamerthemeCambridgeUS}
    %\usepackage{beamerthemeCopenhagen} 
    %\usepackage{beamerthemeDarmstadt}
    %\usepackage{beamerthemeDresden}
    %\usepackage{beamerthemeFrankfurt}
    %\usepackage{beamerthemeGoettingen}
    %\usepackage{beamerthemeHannover}
    \usepackage{beamerthemeIlmenau}
    %\usepackage{beamerthemeJuanLesPins} 
    %\usepackage{beamerthemeLuebeck} 
    \usepackage{pgf,pgfarrows,pgfnodes,pgfautomata,pgfheaps,pgfshade}
    \usepackage{clrscode3e}
    
%     \usepackage[table,xcdraw]{xcolor}

    \beamertemplatetransparentcovereddynamic
    \beamertemplateballitem
    %\beamertemplatefootpagenumber
}
\mode<handout>{
    \usepackage[bar]{beamerthemetree}
    % Colocando um fundo cinza quando for gerar transparências para serem impressas
    % mais de uma transparência por página
    \beamertemplatesolidbackgroundcolor{black!5}
}
% \usepackage[ruled,chapter]{algorithm}
\usepackage{algorithm}	
\usepackage{algorithmic}

\usepackage{comment}
\usepackage{amsmath,amssymb}
\usepackage[brazil]{varioref}
\usepackage[english,brazil]{babel}
\usepackage[utf8]{inputenc}
\usepackage{graphicx}
\usepackage[alf]{abntex2cite}
\usepackage{abntex2abrev}
% \usepackage[round,sort,authoryear,nonamebreak]{natbib}
% \usepackage[authoryear]{natbib}
%\usepackage{minitoc-hyperref}
%\usepackage{listings}
%\usepackage{listings}
%\usepackage{colortbl}
%\usepackage{pslatex}
\beamertemplatetransparentcovereddynamic

\usepackage{tikz}
\usetikzlibrary{arrows}
\tikzstyle{block}=[draw opacity=0.7,line width=1.4cm]

\newcommand*\oldmacro{}%
\let\oldmacro\insertshorttitle%
\renewcommand*\insertshorttitle{%
  \oldmacro\hfill%
  \insertframenumber\,/\,\inserttotalframenumber}

\title[Trabalho de Conclusão de Curso]{Trabalho de Conclusão de Curso}
\subtitle{Desenvolvimento de um sistema web para a Associação dos Protetores da Bacia Hidrográfica do Rio Gorutuba ``Kuruatuba''}

%\titlegraphic{}

\author[Guilherme Rocha Leite]{Guilherme Rocha Leite}
  \institute[UFVJM]{Universidade Federal dos Vales do Jequitinhonha e Mucuri \newline
  Bacharelado em Sistemas de Informação \newline
%   \inst{Departamento de Computação
	  
     Orientador: Prof. Erinaldo Barbosa da Silva\\
     Coorientador: Thales Francisco Mota Carvalho\\
     $~$\\
%      Disciplina: Nome da Disciplina\\
}
\logo{\includegraphics[scale=0.1]{logo-ufvjm.jpg}}
% Se comentar a linha abaixo, irá aparecer a data quando foi compilada a apresentação
%\date{\textcolor{red}{III Encontro Mineiro de Equações Diferenciais, 2009}}
%\pgfdeclareimage[height=0.4cm]{das}{figs/logodas}
%\pgfdeclareimage[height=1cm]{logo}{logos}
% pode-se colocar o LOGO assim
%\logo{\pgfuseimage{logo}}
% ou...
%\logo{\vbox{\hbox to 3cm{\hfil\pgfuseimage{logo}}}}

\begin{document}
%\xdefinecolor{MyGreen}{rgb}{0,0.6,0}
%\beamerboxesdeclarecolorscheme{MyGreen}{MyGreen}{MyGreen!20!averagebackgroundcolor}
\beamerboxesdeclarecolorscheme{formula}{white}{blue!250!averagebackgroundcolor}
\frame{\titlepage}

%\frame{\titlepage}
\part{Presentation}
% \section{Sumário}

\frame{
\frametitle{Sumário}
\tableofcontents
}

\AtBeginSection[]{
  \frame<handout:0>{
    %\frametitle{Sumário}
    \tableofcontents[current,currentsection]
  }
}

%===================================Slide=================================================

\section{Introdução}

\begin{frame}
    \frametitle{Sobre a Kuruatuba}
    \begin{itemize}
        \item \textbf{Fundação:} 1989 - Associação de Futebol de Praia do Copo Sujo de Janaúba;
        \item \textbf{Objetivo:} promoção de esporte, lazer, cultura e preservação e conservação da Bacia do Rio Gorutuba;
        \item \textbf{Parcerias:} Secretaria de Meio Ambiente da Prefeitura de Janaúba e Ruralminas, IEF, CODEMA, Poder Judiciário (Albergados), escolas, igrejas e outros segmentos;
        \item \textbf{Criação do estatuto:} 2003.
    \end{itemize}

\end{frame}


%pag 20
%justificativa 
\begin{frame}
    \frametitle{Motivação}
    \textbf{Principais dificuldades:}
    \begin{itemize}
     \item Impulsionar publicações e atrair apoiadores;
     \item Gerenciar pessoas e arquivos digitais relacionadas à associação. 
    \end{itemize}

    
    \textbf{Problemas encontrados:}
    \begin{itemize}
        \item Informalidade em utilização de blog: opiniões pessoais para assuntos específicos (CENTENO, 2017);
        \item Baixos controle de usuários e personalização visual de um blog; %não tem tanto nível hierárquico entre administradores
        \item Baixa capacidade de atingir o público alvo em publicações; %apenas 7,1% recebem notícias via site ou blog e 56,3% gostariam
        \item Dificuldade em manter arquivos, membros e associados registrados e atualizados.
    \end{itemize}

\end{frame}


%pag 21
\begin{frame}
    \frametitle{Objetivos}
    \textbf{Objetivo geral:} construção de um sistema web para auxiliar na gestão das informações e propagação de conteúdo de autoria da Kuruatuba.
    
    \textbf{Objetivos específicos:}
    \begin{itemize}
        \item Oferecer manutenção, segurança e disponibilidade das informações;
        \item Promover divulgação de vários tipos de conteúdo de maneira organizada;
        \item Possibilitar o gerenciamento de pessoas e administradores vinculados à Kuruatuba;
        \item Aprofundar estudos sobre engenharia e desenvolvimento de software e segurança de dados.
    \end{itemize}
\end{frame}


%pags 24 e 28
%Pra que CMS e Metodologias de desenvolvimento
\section{Desenvolvimento}
\begin{frame}
    \frametitle{Escolha de métodos e ferramentas}
    \textbf{Motivos para usar alguma metodologia ágil:} (SILVA; SOUZA; CAMARGO, 2013)
    \begin{itemize}
        \item Tempo: Cronograma orientado a produto com entregas incrementais (entregas por partes); %vc mostra o cliente em partes, recebe um feedback mais rápido e aplica as alterações a serem apresentadas na próxima sprint
        \item Necessidade de pouca documentação (e consequentemente menos tempo gasto); 
        \item Definição inicial: tempo em \textit{sprints}.
    \end{itemize}
    
\end{frame}

\begin{frame}
 \frametitle{Escolha de métodos e ferramentas}
 %pag 41
    \textbf{Motivos para a escolha do \textit{Scrum}:} (ANWER et al., 2017)
    \begin{itemize}
        \item Elicitação de requisitos: Não definido (à escolha da equipe); %fica a critério da equipe como coletar os requisitos
        \item Ordem de desenvolvimento definida pela equipe \textit{Scrum}; %a equipe tem liberdade de definir quais serão as etapas de desenvolvimento e o prazo para cada uma
        \item Tamanho da equipe: de 1 a 10 indivíduos (PAHUJA (2015); ANWER et al. (2017)); %ficando possível ser realizado com 1 indivíduo
        \item Maioria dos produtores de software aplicam o \textit{Scrum} como metodologia padrão (BEGEL; NAGAPPAN, 2007);
        \item \textit{Framework} muito adaptável em relação a coleta de requisitos, documentação do software, tamanho e prioridade de \textit{sprints}, ritos, etc.  
    \end{itemize}
\end{frame}



\begin{frame}
    \frametitle{Escolha de métodos e ferramentas}
    \textbf{Motivos para usar algum Sistema de Gerenciamento de Conteúdo (CMS):} (Meike, Sametinger e Wiesauer (2009); Chagas, Carvalho e Silva (2018))
    \begin{itemize}
        \item Possibilidade de múltiplos usuários gerenciarem um website ou portal simultaneamente;
        \item Redução de erros de publicação; %O CMS contém os campos necessários que devem ser preenchidos para uma publicação qualquer 
        \item Sem necessidade de conhecimento em programação; %usuários podem manusear sem precisar saber programar
        \item Possibilidade de se definir níveis de acesso a usuários por grupo. %é possível definir quais partes do sistema cada usuário poderá gerenciar
    \end{itemize}
\end{frame}

\begin{frame}
    \frametitle{Escolha de métodos e ferramentas}
    %pag 49
    \textbf{Motivos para a escolha do Plone:}
    \begin{itemize}
        \item Apropriado para construção de portais (OPENLEGIS, 2020); %a exemplo do portal da UFVJM e do portal do governo federal
        \item Sem necessidade de instalação de \textit{plug-ins} adicionais; %que, inclusive, podem afetar o desempenho e a manutenibilidade de um sistema por alta quantidade de plug-ins
        \item Familiaridade do autor com o CMS.
    \end{itemize}
\end{frame}


\begin{frame}
    \frametitle{Escolha de métodos e ferramentas}
    
    % Please add the following required packages to your document preamble:
    % \usepackage[table,xcdraw]{xcolor}
    % If you use beamer only pass "xcolor=table" option, i.e. \documentclass[xcolor=table]{beamer}
    \begin{table}[h]
    \centering
    \caption{Comparativo entre o \textit{Plone} e o \textit{WordPress}. Fonte: (ALVES, 2017). Adaptado.}
    \resizebox{8cm}{!}{ %coloca toda a tabela dentro do slide
        \begin{tabular}{l|c|c}
        \multicolumn{1}{c|}{\textbf{Característica}}  & \textit{\textbf{Plone}} & \textit{\textbf{WordPress}} \\
        \hline 
        Desempenho Geral      & 83,33\%                 & 68,14\%                     \\
        Comércio Eletrônico   & 51,85\%                 & 66,67\%                     \\
        Aplicações Integradas & 75,83\%                 & 66,67\%                     \\
        Flexibilidade         & 100,00\%                & 77,78\%                     \\
        Interoperabilidade    & 95,24\%                 & 66,67\%                     \\
        Gerenciamento         & 91,11\%                 & 53,33\%                     \\
        \rowcolor[HTML]{C0C0C0}
        Performance           & 93,33\%                 & 73,33\%                     \\
        Facilidade de Uso     & 86,27\%                 & 68,63\%                     \\
        Suporte               & 93,33\%                 & 95,56\%                     \\
        \rowcolor[HTML]{C0C0C0}
        Segurança             & 82,46\%                 & 56,14\%              \\  
        
        \hline 
        \end{tabular}
    }
    \label{comaparacao-cms}
    \end{table}
\end{frame}



%pag 40
\begin{frame}
    \frametitle{Outras ferramentas utilizadas}
    \textbf{\textit{Docker}:} 
    \begin{itemize}
        \item Surgimento em 2013;
        \item Estrutura em imagens, camadas e \textit{containers};
        \item Vantagens: escalabilidade (fragmentação de todo o sistema em partes), alto desempenho, portabilidade (facilidade em executar o sistema em ambientes diferentes), verificação de erros via relatório de \textit{logs} e isolamento de \textit{containers} à máquinas externas. 
        
%         O container do MySQL fica mais seguro por impedir que outras máquinas o acessem
% 
%         o processo de migração para outra VPS é muito mais fácil por causa dos comandos (inclusive pra pegar imagens que eu criei)
% 
%         Com docker a aplicação tem um desempenho melhor que em VMs porque eu consigo pegar uma imagem muito específica e adicionar só o que eu quero, além de que o Docker usar recursos (hardware e software) da própria máquina hospedeira, e na VM ela tem recursos próprios dela

    \end{itemize}
    
\end{frame}

\begin{frame}
    \frametitle{Outras ferramentas utilizadas}
    \textbf{\textit{Git}:} 
    \begin{itemize}
        \item Surgimento em 2005;
        \item Utilizado para versionamento de código; %consegue acompanhar e retornar à versões anteriores
        \item Vantagens: gerenciar versionamento de código, desenvolver aplicações em colaboração e possibilidade de manter código privado quando manuseado com um site de hospedagem de códigos.
    \end{itemize}
\end{frame}


\begin{frame}
    \frametitle{Outras ferramentas utilizadas}
    \textbf{\textit{Google Analytics}:} 
    \begin{itemize}
        \item Objetivo: obtenção de relatório com dados públicos sobre visitantes de um website;
        \item Vantagens: identificação de padrões relacionados a visitantes, como origem e número de acessos, conteúdos acessados e tempo de duração de cada sessão. %sendo um grande medidor do impacto dos conteúdos publicados e seu alcance
    \end{itemize}
\end{frame}


\begin{frame}
    \frametitle{Outras ferramentas utilizadas}
    \textbf{MySQL:} 
    \begin{itemize}
        \item Tipo de Sistema Gerenciador de Bancos de Dados (SGBD) que atua com bancos de dados relacionais; %bancos que atuam com relacionamentos entre tabelas
        \item Vantagens: possui todos os atributos necessários a um banco de dados de grande porte (MILANI, 2007), possui código aberto e de fácil aprendizagem e utilização. 
    \end{itemize}
    
\end{frame}



%pag 54
\begin{frame}
    \frametitle{Coleta e documentação de requisitos}
    \textbf{Recursos utilizados:}
    \begin{itemize}
        \item Reuniões e questionários;
        \item Histórias de usuário;
        \item Diagramas de casos de uso;
        \item Fluxos de eventos.
    \end{itemize}
\end{frame}

\begin{frame}
    \frametitle{Requisitos obtidos} %importante mencionar que como foi usado o Scrum, outras solicitações foram feitas ao entrar em contato com o cliente
    \begin{enumerate}
     \item Sistema de \textit{login} para os usuários;
     \item Área para publicação de notícias, eventos e mais informações sobre a associação;
     \item Cadastro, atualização e remoção de associados;
     \item Sistema para gerar carteirinha de vínculo para associados;
     \item Formulário para recebimento de mensagens dos visitantes;
     \item \textit{Link} para \textit{download} de documentos importantes da Kuruatuba.
    \end{enumerate}

\end{frame}


%pag 56
\begin{frame}
    \frametitle{Diagrama de casos de uso do portal da Kuruatuba}
    \begin{figure}[htb]
        \centering
        
        \includegraphics[width=0.7\textwidth]{figuras/use-case-portal-1.png}
        
        \label{use-case-portal}
    \end{figure}
\end{frame}

%pag 56
\begin{frame}
    \frametitle{Diagrama de casos de uso do portal da Kuruatuba}
    \begin{figure}[htb]
        \centering
        
        \includegraphics[width=0.6\textwidth]{figuras/use-case-portal-2.png}
        
        \label{use-case-portal}
    \end{figure}
\end{frame}

\begin{frame}
    \frametitle{Diagrama de casos de uso do sistema de associados da Kuruatuba}
    \begin{figure}[htb]
        \centering
        \includegraphics[width=0.56\textwidth]{../figuras/use-case-sistema.png}
        
        \label{use-case-sistema}
    \end{figure}
\end{frame}


%Só se achar que tem tempo pra falar:
%pag 74
% \begin{frame}
%     \frametitle{\textit{Product Backlog}}
%     \begin{table}[h]
%         \centering
%         \caption{\textit{Product Backlog}. Fonte: Autor.}
%         
%         \resizebox{\columnwidth}{!}{ %coloca toda a tabela dentro do slide
%             \begin{tabular}{c|c|c|c}
%                 
%                 \textbf{Nome} & \textbf{Tarefas} & \textbf{Prioridade} & \textbf{Prazo (dias)} \\ % Note a separação de col. e a quebra de linhas
%                 \hline                               % para uma linha horizontal
%                 
%                 \textit{Sprint 1} & Criação da interface inicial & 5 & 14  \\
%                 \rule{0pt}{13pt}
%                 \textit{Sprint 2} & Contratação e configuração do ambiente de desenvolvimento & 5 & 11  \\ 
%                 \rule{0pt}{13pt}
%                 \textit{Sprint 3} & Migração do portal e do sistema de associados para o servidor & 5 & 6  \\
%                 \rule{0pt}{13pt}
%                 \textit{Sprint 4} & Configuração do banco de dados e sua conexão com o sistema & 4 & 3  \\
%                 \rule{0pt}{13pt}
%                 \textit{Sprint 5} & Criação do sistema de autenticação com criptografia & 2 & 5  \\
%                 \rule{0pt}{13pt}
%                 \textit{Sprint 6} & Organização da página inicial do portal & 5 & 6  \\
%                 \rule{0pt}{13pt}
%                 \textit{Sprint 7} & Inserção de informações nas páginas & 5 & 20  \\ 
%                 \rule{0pt}{13pt}
%                 \textit{Sprint 8} & Ajustes no CRUD de associados e administradores & 3 & 6  \\
%                 \rule{0pt}{13pt}
%                 \textit{Sprint 9} & Criação do sistema de recuperação de senha e do \textit{slider} da galeria & 3 & 5  \\
%                 \rule{0pt}{13pt}
%                 \textit{Sprint 10} & Configuração do \textit{Google Analytics} e ajustes gerais & 2 & 6 \\
%                 
%                 \hline
%             \end{tabular}
%         }
%         \label{lista-sprints}
%     \end{table}
% \end{frame}


%Só se achar que tem tempo pra falar:
%pag 76
% \begin{frame}
%     \frametitle{Estrutura de \textit{containers} do sistema}
%     \begin{itemize}
%         \item \textbf{Docker}
%         \item \textbf{Git}
%     \end{itemize}
% \end{frame}


%pag 79
\section{Resultados}
\begin{frame}
    \frametitle{Apresentação das Telas: Tela de Login do Portal}
    \begin{figure}[htb]
        \centering
        \includegraphics[width=0.9\textwidth]{../figuras/kuruatuba_portal_login.png}
        \label{fig:login-portal}
    \end{figure}
\end{frame}

\begin{frame}
    \frametitle{Apresentação das Telas: Tela de Login do Sistema de Associados}
    \begin{figure}[htb]
        \centering
        \includegraphics[width=0.9\textwidth]{../figuras/kuruatuba_sistema_login.png}
        \label{fig:login-sistema}
    \end{figure}
\end{frame}

\begin{frame}
    \frametitle{Apresentação das Telas: Página Inicial do Sistema de Associados}
    \begin{figure}[htb]
        \centering
        \includegraphics[width=0.9\textwidth]{../figuras/kuruatuba_sistema_home.png}
        \label{fig:home-sistema}
    \end{figure}
\end{frame}



%pag 84
\section{Discussão}
\begin{frame}
    \frametitle{Discussão}
    \begin{itemize}
        \item Cumprimento de todos os objetivos;
        \item Cumprimento das exigências definidas nos requisitos;
        \item Implementação de métodos contra ataques de SQL Injection e Sequestro de Sessão; %não que enibem totalmente, mas dificultam ações dessa natureza
        \item Implementação de métodos que previnem ações danosas e involuntárias. %como: duplicidade de registros em banco de dados, cadastro de informações pessoais inválidas e exclusão involuntária.
    \end{itemize}
\end{frame}


\begin{frame}
    \frametitle{Dificuldades Encontradas}
    \begin{itemize}
     \item Coleta de informações para preenchimento de páginas;
     \item Escolha dos arquivos a serem enviados para a galeria do portal;
     \item Desempenho dos papéis do \textit{Scrum}; %difícil analisar em tempo real quais atividades estão atribuídas a qual papel e seguir os ritos comuns, como: cunsulta constante ao cliente, análise de impedimentos para realizar as funções
     \item Configuração da VPS e do apontamento domínio/hospedagem.
    \end{itemize}

\end{frame}



%pag 87
\section{Trabalhos futuros}
\begin{frame}
    \frametitle{Trabalhos futuros}
    \begin{itemize}
        \item Criar manual de utilização do sistema e executar treinamento com futuros usuários;
        \item Instalar um certificado digital (SSL); %em pelo menos as telas de login
        \item Executar testes de software a fim de se obter feedback sobre integridade, disponibilidade e acessibilidade. %Isso é opcional
    \end{itemize}

    
\end{frame}

    

\section{Referências Bibliográficas}

\begin{frame}
    \frametitle{Referências Bibliográficas}
    \scriptsize{
        ALVES, E. d. C. Reconstrução do portal institucional da UFVJM: adoção da Identidade Digital do Governo Federal e implementação do PloneGov-BR como novo Sistema de Gerenciamento de Conteúdo — Universidade dos Vales do Jequitinhonha e Mucuri, Diamantina, MG, Brasil, 2017. \newline

        ANWER, F. et al. Comparative analysis of two popular agile process models: Extreme programming and scrum. International Journal of Computer Science and Telecommunications, v. 8, n. 2, p. 1–7, 2017. \newline

        BEGEL, A.; NAGAPPAN, N. Usage and perceptions of agile software development in an industrial context: An exploratory study. In: IEEE. First International Symposium on Empirical Software Engineering and Measurement (ESEM 2007). [S.l.], 2007. p. 255–264. \newline

        CENTENO, D. B. Pampatur blog. Universidade Federal do Pampa, 2017. \newline
        
        CHAGAS, F.; CARVALHO, C. L. de; SILVA, J. C. da. Um estudo sobre os sistemas de gerenciamento de conteúdo de código aberto. Revista Telfract, v. 1, n. 1, 2018. \newline
}
\end{frame}

\begin{frame}
    \frametitle{Referências Bibliográficas}
    \scriptsize{
        %Essa referência dá erro, só não sei porque
        MEIKE, M.; SAMETINGER, J.; WIESAUER, A. Security in open source web content management systems. IEEE Security & Privacy, IEEE, v.7, n. 4, p.44-51, 2009. \newline

        MILANI, A. MySQL-guia do programador. [S.l.]: Novatec Editora, 2007. \newline

        OPENLEGIS. Portal Legislativo — OpenLegis. Disponível em: https://www.openlegis.com.br/servicos/portal-legislativo/. Acesso em: 11 de set. de 2021. \newline
        
        PAHUJA, S. Scrum for Individuals, 2015. Disponível em: https://www.infoq.com/news/2015/02/personal-scrum/. Acesso em: 12 de set. de 2021. \newline

        SILVA, D. E. dos S.; SOUZA, I. T. de; CAMARGO, T. Metodologias ágeis para o desenvolvimento de software: Aplicação e o uso da metodologia scrum em contraste ao modelo tradicional de gerenciamento de projetos. Revista Computação Aplicada-UNG-Ser, v. 2, n. 1, p. 39–46, 2013. \newline
    }
\end{frame}


\end{document}
